\chapter{第二階段}
\section{第一部分:描述家人,談論過去發生的事情}
\begin{itemize}
    \item \textbf{重點語句\textemdash 描述家人}\\
        \indent ご\ruby{両|親}{りょ|しん}は\ruby{元|気}{げん|き}ですか?\ruby{父}{ちち}も\ruby{母}{はは}も元気です。
        姉も兄も日本にいます。\\
        弟は小学三年生です。\\
        妹はお肉が好きじゃないです。\\
        弟さんはお幾つですか?弟は五歳です。\par
        \textbf{語法點:父 VS. お父さん}\\
        之前我們講到使用類似於\underline{お\ruby{父}{とう}さん}或\underline{お\ruby{母}{かあ}さん}的單詞來跟自己的家庭成員交談或談論別人的家庭。但當你跟別人介紹自己家時,用的卻是不同的稱呼。
        \begin{center}
            \begin{tabular}{@{}ll@{}}
                \toprule
                自己的家人 & 別人的家人\\
                \midrule
                \ruby{父}{ちち} & お\ruby{父}{とう}さん\\
                \ruby{母}{はは} & お\ruby{母}{かあ}さん\\
                \ruby{兄}{あに} & お\ruby{兄}{にい}さん\\
                \ruby{姉}{あね} & お\ruby{姉}{ねえ}さん\\
                \ruby{弟}{おとうと} & \ruby{弟}{おとうと}さん\\
                \ruby{妹}{いもうと} & \ruby{妹}{いもうと}さん\\
                \bottomrule
            \end{tabular}
        \end{center}
    \item \textbf{重點語句\textemdash 談論過去發生的事情}\\
        昨日は魚お食べましたか?\\
        昨日は何も食べませんでした。\\
        \ruby{一昨日}{おととい}はニナをしましたか?\\
        一昨日は朝ご飯を食べませんでした。\par
        \textbf{語法點}\\
        過去就用過去式!\\
        當你使用日語的過去式,可以將動詞中的\underline{\textendash ます}換成\underline{\textendash ました}。
        \begin{eg}\leavevmode
            \begin{itemize}
                \item \ruby{勉|強}{べん|きょう}します。(我)學習。
                \item 勉強し\underline{ました}。(我)學習了。
            \end{itemize}
        \end{eg}
        要表達過去沒有做的事情,將\underline{\textendash ません}換成\underline{\textendash ませんでした}就可以啦!
        \begin{eg}\leavevmode
            \begin{itemize}
                \item 勉強しません。(我)不學習。
                \item 勉強し\underline{ませんでした}。(我)沒學習。
            \end{itemize}
        \end{eg}
\end{itemize}
\expandafter\string\the\font
