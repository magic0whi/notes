\chapter{Dynamics of First-Order Difference Equations}
\section{Introduction}
\begin{itemize}
    \item Starting from $x_0$, we have first iterate $f(x_0)$ whereas second iterate $f(f(x_0))=f^2(x_0)$; \\
        more generally $n$th iterate $f^n(x_0)$. Especially $f^0(x_0)=x_0$.
    \item The set of all (positive) iterates $\{f^n(x_0):n\geq 0\}=O(x_0)$, called (positive) orbit of $x_0$.
    \item $x(n+1)$ is a function of the $n$th iterate number $x(n)$:
        \[\text{Autonomous, time-invariant system }x(n+1)=f^{n+1}(x_0)=f(x(n)).\]
    \item $x(n+1)$ might also have this form. (Here $g: \mathbb{Z}^+ \times\mathbb{R}\to\mathbb{R}$)
        \[\text{Nonautonomous, time-variant system }x(n+1)=g(n,x(n))\]
        If an initial condition $x(n_0)=x_0$ is given, then for $n\geq n_0$, there is a \textit{unique solution} $x(n)\equiv x(n,n_0,x_0)$ of nonautonomous $x(n+1)=g(n,x(n))$
        \begin{explanation}
            This may be shown easily by iteration:
            \[
            \begin{array}{l}
                x(n_0+1)=g(n_0,x(n_0))=g(n_0,x_0) \\
                x(n_0+2)=g(n_0+1,x(n_0+1))=g(n_0+1,g(n_0,x_0)) \\
                x(n_0+3)=g(n_0+2,x(n_0+2))=g[n_0+2,g(n_0+1,g(n_0,x_0))]
            \end{array}
        \]
        Inductively, we get $x(n)=g[n-1,x(n-1)]=x(n,n_0,x_0)$ because there is unknown $x_0$, $n_0$ in $x(n-1)$.
        \end{explanation}
\end{itemize}

\section{Linear First-Order Difference Equations}
\begin{itemize}
    \item Typical linear \textit{homogeneous} first-order equation:
        \begin{align*} & x(n+1)=a(n)x(n) & \left(x(n_0)=x_0, n\geq n_0\geq 0\right), \end{align*}
        and the associated \textit{nonhomogeneous} equation:
        \begin{align*} & y(n+1)=a(n)y(n)+g(n) & \left(y(n_0)=y_0, n\geq n_0\geq 0\right), \end{align*}
        where in both equations it is assumed that $a(n)\neq 0$, and $a(n)$ and g(n are real-valued functions defined for $n\geq n_0\geq 0$.
    \item The solution of $x(n+1)=a(n)x(n)$:
        \[x(n)=x_0\left[\prod_{i=n_0}^{n-1} a(i)\right]\]
        \begin{explanation}
            May be obtained by a simple iteration:
            \[
            \begin{array}{l}
                x(n_0+1)=a(n_0)x(n_0)=a(n_0)x_0 \\
                x(n_0+2)=a(n_0+1)x(n_0+1)=a(n_0+1)a(n_0)x_0 \\
                x(n_0+3)=a(n_0+2)x(n_0+2)=a(n_0+2)a(n_0+1)a(n_0)x_0
            \end{array},
        \]
        inductively it is easy to see that $x(n)=a(n-1)a(n-2)\cdots a(n_0)x_0$
        \end{explanation}
        The unique solution of $y(n+1)=a(n)y(n)+g(n)$:
        \[y(n)=y_0\left[\prod_{i=n_0}^{n-1} a(i)\right]+\sum_{r=n_0}^{n-1}\left[\prod_{i=r+1}^{n-1} a(i)\right]g(r)\]
        \begin{explanation}
            This may be found by iteration and proof by mathematical induction. Also notice that $\prod_{i=n}^{n-1} a(i)=1$.
            \begin{align*}
                y(n_0+1) & =a(n_0)y_0+g(n_0) \\
                y(n_0+2) & =a(n_0+1)y(n_0+1)+g(n_0+1) \\
                         & =a(n_0+1)a(n_0)y_0+a(n_0+1)g(n_0)+g(n_0+1) \\
                y(n_0+3) & =a(n_0+2)y(n_0+2)+g(n_0+2) \\
                         & =a(n_0+2)a(n_0+1)a(n_0)y_0+a(n_0+2)a(n_0+1)g(n_0) \\
                         & \quad +a(n_0+2)g(n_0+1)+g(n_0+2)
            \end{align*}
            Inductively we can see the unique solution. To establish this, assume this formula holds for $n=k$, lets substitute into its original nonhomogeneous equation:
                \begin{align*}
                    y(k+1) & =a(k)y_0\left[\prod_{i=n_0}^{k-1} a(i)\right]+\sum_{r=n_0}^{k-1}\left[a(k)\prod_{i=r+1}^{k-1} a(i)\right]g(r)+g(k) \\
                           & =y_0\left[\prod_{i=n_0}^k a(i)\right]+\sum_{r=n_0}^{k-1}\left[\prod_{i=r+1}^k a(i)\right]g(r)+\left[\prod_{i=k+1}^k a(i)\right]g(k) \\
                           & =y_0\left[\prod_{i=n_0}^k a(i)\right]+\sum_{r=n_0}^k\left[\prod_{i=r+1}^k a(i)\right]g(r)
                \end{align*}
            Hench the unique solution holds for all $n\in\mathbb{Z}^+$
        \end{explanation}
\end{itemize}

\subsection{Important Special Cases}
\begin{enumerate}
    \item There are two special cases of $y(n+1)=a(n)y(n)+g(n)$, one of them is
        \begin{align*} & y(n+1)=ay(n)+g(n) & \left(y(0)=y_0\right) \end{align*}
        Using the solution formula may establish that
        \[y(n)=y_0a^n+\sum_{k=0}^{n-1} a^{n-k-1}g(k)\]
    \item The another one is
        \begin{align*} & y(n+1)=ay(n)+b & \left(y(0)=y_0\right) \end{align*}
        Using the unique solution we obtain
        \[
            y= \begin{cases}
                y_0a^n +b\frac{a^n-1}{a-1} & \text{if }a\neq 1, \\
                y_0+bn                    & \text{if }a=1.
            \end{cases}
        \]
        Notice here we use the sum formula of the Geometric Series $S_n=\frac{a_1(1-q^n)}{1-q}$ in the form $\sum_{k=0}^{n-1} a^{n-k-1}=a^{n-1}+\cdots+a^0\quad\left(a_1=a^0=1, q=a\right)$.
    \item The solution of the nonhomogeneous differential equation $\frac{\mathrm{d}y}{\mathrm{d}t}=ay(t)+g(t)\quad\left(y(0)=y_0)\right)$ is given by $y(t)=e^{at}y_0+\int_0^t e^{a(t-s)}g(s)\mathrm{d}s$.
        \begin{explanation}
            The solution uses the complete solution of first-order nonhomogeneous differential equation, which is hard to explain in a calculus courses. However, we can prove it in a forward way by taking derivatives of the solution:
            \begin{align*}
                y^\prime(t)=\frac{\mathrm{d}y}{\mathrm{d}t} & =ae^{at}y_0+\left[e^{at}\int_0^t e^{-as}g(s)\mathrm{d}s\right]^\prime \\
                & =ae^{at}y_0+(e^{at})^\prime\int_0^t e^{-as}g(s)\mathrm{d}s+e^{at}\left[\int_0^t e^{-as}g(s)\mathrm{d}s\right]^\prime \\
                & =ae^{at}y_0+ae^{at}\int_0^t e^{-as}g(s)\mathrm{d}s+e^{at}\underbrace{\left[\int_0^t e^{-as}g(s)\mathrm{d}s\right]^\prime}_{\begin{subarray}{l}
                        \because F^\prime(x)=f(x)=e^{-ax}g(x) \\
                        \therefore =[F(t)-F(0)]^\prime \\
                        \enspace=F^\prime(t)-F^\prime(0) \\
                        \enspace=e^{-at}g(t)
                \end{subarray}} \\
                & =ae^{at}y_0+ae^{at}\int_0^t e^{-as}g(s)\mathrm{d}s+e^{at}[e^{-at}g(t)] \\
                & =ae^{at}y_0+ae^{at}\int_0^t e^{-as}g(s)\mathrm{d}s+g(t) \\
                & =a\left[e^{at}y_0+\int_0^t e^{a(t-s)}g(s)\mathrm{d}s\right]+g(t) \\
                & =ay(t)+g(t)
            \end{align*}
        \end{explanation}
\end{enumerate}

\textbf{Examples}

\begin{enumerate}
    \item Solve the equation
        \begin{align*} & y(n+1)=(n+1)y(n)+2^n(n+1)! & \left(y(0)=1,n\ge 0\right) \end{align*}
        \begin{remark}
			\begin{tabular}{ccc}
			\multicolumn{3}{c}{Definite sum.} \\ \hline
			Number & Summation & Definite sum \\ \hline
            1      & $\sum_{k=1}^n k$ & $\frac{n(n+1)}{2}$ \\
            2      & $\sum_{k=1}^n k^2$    & $\frac{n(n+1)(2n+1)}{6}$ \\
            3      & $\sum_{k=1}^n k^3$    & $\left[\frac{n(n+1)}{2}\right]^2$ \\
            4      & $\sum_{k=1}^n k^4$    & $\frac{n(6n^4+15n^3+10n^2-1)}{30}$ \\
            5      & $\sum_{k=0}^{n-1} a^k$ & $\begin{cases}
                                            (a^n-1)/(a-1) & \text{if }a\neq 1 \\
                                            n & \text{if }a=1
                                          \end{cases}$ \\
            6      & $\sum_{k=1}^{n-1} a^k$ & $\begin{cases}
                                            (a^n-a)/(a-1) & \text{if }a\neq 1 \\
                                            n-1 & \text{if }a=1
                                          \end{cases}$ \\
            7      & $\sum_{k=1}^n ka^k\quad\left(a\neq 1\right) $ & $\frac{(a-1)(n+1)a^{n+1}-a^{n+2}+a}{(a-1)^2}$ \\ \hline
			\end{tabular}
        \end{remark}
        Solution
        \begin{align*}
            y(n) & =\prod_{i=0}^{n-1} (i+1)+\sum_{k=0}^{n-1}\left[\prod_{i=k+1}^{n-1} (i+1)\right]2^k(k+1)! \\
                 & =n!+\sum_{k=0}^{n-1} n!2^k \\
                 & =2^nn!\quad\text{(see the table above)}
        \end{align*}
    \item Find a solution for the equation
        \begin{align*}
            x(n+1)=2x(n)+3^n & \left(x(1)=0.5\right).
        \end{align*}
        Solution\quad From the unique solution of nonhomogeneous equation and of special case 1, we have
        \begin{align*}
            x(n) & =2^{n-1}\cdot0.5+\sum_{k=1}^{n-1}2^{n-k-1}3^k \\
                 & =2^{n-2}+2^{n-1}\sum_{k=1}^{n-1}(\frac{3}{2})^k \\
                 & =2^{n-2}+2^{n-1}\left(\frac{(\frac{3}{2})^n-\frac{3}{2}}{\frac{3}{2}-1}\right) \\
                 & =2^{n-2}+2^{n-1}\frac{3}{2}\left(\frac{(\frac{3}{2})^{n-1}-1}{\frac{3}{2}-1}\right) \\
                 & =2^{n-2}\left(1+3\frac{(\frac{3}{2})^{n-1}-1}{\frac{3}{2}-1}\right) \\
                 & =2^{n-2}\left(\frac{\frac{3}{2}-1}{\frac{3}{2}-1}+\frac{3^n\cdot 2^{1-n}-3}{\frac{3}{2}-1}\right) \\
                 & =2^{n-2}\left(\frac{3^n\cdot 2^{1-n}-\frac{5}{2}}{\frac{3}{2}-1}\right) \\
                 & =\frac{\frac{3^n}{2}-2^{n-2}\frac{5}{2}}{\frac{3}{2}-1} \\
                 & =\frac{3^n-2^{n-2}\cdot 5}{2-1} \\
                 & = 3^n-5\cdot 2^{n-2}.
        \end{align*}
    \item A drug is administered once every four hours. Let $D(n)$ be the amount of the drug in the blood system at the $n$th interval. The body eliminates a certain fraction $p$ of the drug during each time interval. If the amount administered is $D_0$, find $D(n)$ and $\lim_{n\to\infty} D(n)$.

        Solution\quad We first must create an equation to solve. Since the amount of drug in the patient's system at time $(n+1)$ is equal to the amount at time $n$ minus the fraction $p$ that has been eliminated from the body, plus the new dosage $D_0$, we arrive at the following equation:
        \[D(n+1)=(1-p)D(n)+D_0.\]
        Using the solution of special case 2, arriving at
        \begin{align*}
            D(n) & =(1-p)^nD_0+D_0\frac{(1-p)^n-1}{-p} \\
                 & =\left[D_0-\frac{D_0}{p}\right](1-p)^n+\frac{D_0}{p}
        \end{align*}
        Hence,
        \[\lim\limits_{n\to\infty} D(n)=\frac{D_0}{p}.\]
    \item Amortization

        Amortization is the process by which a loan is repaid by a sequence of periodic payments, each of which is part payment of interest and part payment to reduce the outstanding principal.

        Let $p(n)$ represent the outstanding principal after the $n$th payment $g(n)$. Suppose that interest charges compound at the rate $r$ per payment period.

        The formulation of our model here is based on the fact that the outstanding principal $p(n+1)$ after the $(n+1)$st payment is equal to the outstanding principal $p(n)$ after the $n$th payment plus the interest $rp(n)$ incurred during the $(n+1)$st period minus the $n$th payment $g(n)$. Hence
        \begin{align*}
            p(n+1) & =p(n)+rp(n)-g(n) & \left(p(0)=p_0\right) \\
                   & =(1+r)p(n)-g(n) &
        \end{align*}
        where $p_0$ is the initial debt. By solution of special case 1 we have
        \[p(n)=(1+r)^np_0-\sum_{k=0}^{n-1} (1+r)^{n-k-1}g(k)\]
        In practice, the payment $g(n)$ is constant and, say, equal to $T$. from solution of special case 2 we have
        \[p(n)=(1+r)^np_0-[(1+r)^n-1](\frac{T}{r}).\]
        If we want to pay off the loan in exactly $n$ payments, we have $p(n)=0$, the monthly payment
        \[T=p_0\left[\frac{r}{1-(1+r)^{-n}}\right].\]
\end{enumerate}

\section{Equilibrium Points}
\begin{definition}
    A point $x^*$ in the domain of $f$ is said to be an \textit{equilibrium point} of $x(n+1)=f(x(n))$ if it is a fixed point of $f$, i.e., $f(x^*)=x^*$
\end{definition}

In other words, $x^*$ is a \textit{constant solution} of $x(n+1)=f(x(n))$, since if $x(0)=x^*$ is an initial point, then $x(1)=f(x^*)=x^*$, and $x(2)=f(x(1))=f(x^*)=x^*$, and so on.

Graphically, an equilibrium point is the $x$-coordinate of the point where the graph of $f$ intersects the diagonal line $x=y$ (Figures 1.1 and 1.2). For example, there are three equilibrium points for the equation
\[x(n+1)=x^3(n)\]
where $f(x)=x^3$. To find these equilibrium points, we let $f(x^*)=x^*$, or $x^3=x$, or $x^3=x$, and solve for $x$. Hence there are three equilibrium points, $-1$, $0$, $1$ (Figure 1.1).

Figure 1.2 illustrates another example, where $f(x)=x^2-x+1$ and the difference equation is given by
\[x(n+1)=x^2(n)-x(n)+1.\]
Letting $x^2-x+1=x$, we find that $1$ is the only equilibrium point.

\begin{definition}
    Let $x$ be a point in the domain of $f$. If there exists a positive integer $r$ and an equilibrium point $x^*$ of $x(n+1)=f(x(n))$ such that $f^r(x)=x^*$, $f^{r-1}(x)\neq x^*$, then $x$ is an \textit{eventually equilibrium (fixed) point}.
\end{definition}

\begin{eg}
    \item The Tent Map
        Consider the equation (Figure 1.3)
        \[x(n+1)=T(x(n))\]
        where
        \[
            T(x)=\begin{cases}
                2x & \text{for }0\leq x\leq\frac{1}{2}, \\
                2(1-x) & \text{for }\frac{1}{2}<x<1
            \end{cases}
        \]
        There are two equilibrium points, $0$ and $\frac{2}{3}$ (see Figure 1.3). The search or eventually equilibrium points is not as simple algebraically. If $x(0)=\frac{1}{4}$, then $x(1)=\frac{1}{2}$, $x(2)=1$, and $x(3)=0$. Thus $\frac{1}{4}$ is an eventually equilibrium point.
\end{eg}

One of the main objectives in the study of a dynamical system is to analyze the behavior of its solutions near an equilibrium point. This study constitutes the stability theory.

\begin{definition}
    (a) The equilibrium point $x*$ of $x(n+1)=f(x(n))$ is \textit{stable} (Figure 1.4) if given $\varepsilon>0$ there exists $\delta>0$ such that $|x_0-x^*|<\delta$ implies $|f^n(x_0)-x^*|<\varepsilon$ for all $n>0$. If $x^*$ is not stable, then it is called \textit{unstable} (Figure 1.5).

    (b) The point $x^*$ is said to be \textit{attracting} if there exists $\eta>0$ such that
    \[|x(0)-x^*|<\eta\text{ implies }\lim_{n\to\infty} x(n)=x^*.\]
    If $\eta=\infty$, $x^*$ is called a \textit{global attractor} or \textit{globally attracting}
    
    (c) The point $x^*$ is an \textit{asymptotically stable equilibrium point} if it is stable and attracting.

    If $\eta=\infty$, $x^*$ is said to be \textit{globally asymptotically stable} (Figure 1.7).
\end{definition}
