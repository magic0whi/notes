\lesson{1}{Nov 10 2023 16:00}{}

\chapter{Sketch graph of factored polynomial function}
\begin{itemize}
    \item Multiplicity (Order): Number of times a give factor appears in the factored polynomial. If a polynomial contains a factor $(x-h)$, then $x=h$ is called zero of multiplicity $p$.
    \item Degree of a polynomial: Highest degree of a term in a polynomial, which is determined by the number of variable factor in that term. It is also the sum of multiplicities.
        \begin{eg}
            \[
                f(x)=-\frac{2}{5}x^7 y^2 z+\sqrt{3}x^2 yz-x+2\pi
            \]
        \end{eg}
    \item Concavity
        \begin{itemize}
                \item If a line segment joining any two points on a curve is entirely below the line, then the curve is said to be \textbf{concave down} between the two points.
                    \begin{figure}[H]
                        \centering
                        \incfig{1}
                        \caption{Concave down}
                    \end{figure}
                \item If a line segment is entirely above the curve, the curve is \textbf{concave up} between the two points.
                    \begin{figure}[H]
                        \centering
                        \incfig{2}
                        \caption{Concave up}
                    \end{figure}
        \end{itemize}
    \item Point of inflection: a point where a graph of function changes concavity.
    \item $n$th degree polynomial functions will have a maximum of $n-1$ turning points since it has maximum $n$ $x$-intercepts.
    \item If degree $n$ is odds polynomial functions have at least one $x$-intercept; while the even degree polynomial may have no $x$-intercepts
\end{itemize}

\section{Sketch graph of factored polynomial function}
\begin{enumerate}
    \item Find $x$-intercepts $f(x)=0$ and $y$-intercept $f(0)$.
    \item Check symmetry, whether $f(-x)=f(x)$ or $f(-x)=-f(x)$.
    \item Use multiplicities of zeros  to determine the behavior of polynomial at $x$-intercepts.

        For each factor $(x-h)^p$
        \begin{figure}[H]
            \centering
            \incfig{3}
            \caption{Graph of three typical power functions}
        \end{figure}
        Higher power $p$ appears flatter on the graph..
    \item Determine end behavior by examining the leading term.
        \[
            f(x)=a_n x^n+a_{n-1}x^{n-1}+\cdots+a_1 x+a_0
        \]
        \begin{figure}[H]
            \centering
            \incfig{4}
            \caption{End behavior is depended on leading term}
        \end{figure}
    \item Sketch the graph. Uses the end behavior and the behavior at intercepts.
\end{enumerate}

\section{Even and Odd functions}
\begin{itemize}
    \item A polynomial function is an \textbf{even function} if and only if each of the terms of the function is of an even degree. \\
        Graph of odd degree polynomial functions will never have even symmetry.
    \item A polynomial function is an \textbf{odd function} if and only if each of the terms of the function is of an odd degree.
        Graph of even degree polynomial functions will never have odd symmetry.
\end{itemize}

\section{Division of polynomials}
\begin{itemize}
    \item Long division
        \[
            \begin{array}{r}
                \phantom{2x^2+}2x-1 \\
                x+3\overline{\smash{\big)}2x^2+5x+8} \\
                \underline{2x^2+6x}\phantom{{}+\,}\downarrow \\
                -1x-8 \\
                \underline{-1x-3} \\
                -5
            \end{array}
        \]
    \item Synthetic division
        \[
            \begin{array}{rrrr}
                \multicolumn{1}{r|}{-3} & 2 & 5 & -8 \\
                \multicolumn{1}{r|}{} & \downarrow & -6 & 3 \\ \cline{2-4}
                & 2 & -1 & -5
            \end{array}
        \]
    \item To carry out the division, ensure the terms of the polynomial are in descending order of degree. Missing powers of $x$ in the dividend or divisor are included using a coefficient of 0 to keep work accurate and aligned correctly.
    \item The division is completed when the degree of the remaining terms after subtraction is less than the degree of the divisor.
        \[
            \begin{array}{c@{}c@{}c@{}c@{}c@{}c@{}c}
            P(x) & {}={} & D(x) & Q(x) & + & R(x) & {}\Leftrightarrow \frac{P(x)}{D(x)}=Q(x)+\frac{R(x)}{D(x)} \\
            n & & n-r & r & & <r &
            \end{array}
        \]
    \item The Remainder Theorem

        When a polynomial $P(x)$ is divided by $(mx-n)$, the remainder is $P(\frac{n}{m})$ where $m$ and $n$ are integers and $m\neq 0$.
    \item The Factor Theorem
        
        $(mx-n)$ is a factor of $P(x)\iff P(\frac{n}{m})=0$.
    \item The Rational Root Theorem

        For $a_nx^n+a_{n-1}x^{n-1}+\cdots+a_0=0$, $\exists x=\frac{p}{q}$, $p$ is a factor of $a_0$, $q$ is a factor of $a_n$.
\end{itemize}

\section{Rational Functions 1}
\begin{itemize}
    \item A rational function is a function of the form $f(x)=\frac{g(x)}{h(x)}$, where $g(x)$ and $h(x)$ are polynomials and $h(x)\neq 0$.
    \item The function has a vertical asymptote (V.A.) $x=a$ if $h(a)=0$ and $g(a)\neq 0$.
    \item If $h(a)=0$ and $g(a)=0$ then $x=a$ is a factor of the numerator and denominator of the function, and a hole occur at $x=a$. 
    \item If the degree of $g(x)$ is less than the degree of $h(x)$, then $y=0$ is the horizontal asymptote of the function.
        \[
            x\to\pm\infty,\,y\to0
        \]
    \item If the degree of $g(x)$ is equals the degree of $h(x)$, then $y=\frac{a}{b}$ is the horizontal asymptote of the function, where $a$ and $b$ are the coefficients of the highest degree term in the numerator and denominator, respectively
    \item If the degree of $g(x)$ is greater than the degree of $h(x)$, there is no horizontal asymptote.
    \item If the degree of $g(x)$ is exactly one greater than the degree $h(x)$ then the function has an oblique asymptote (O.A.).
    \[
        y=\frac{q(x)}{h(x)}\iff y=ax+b+\frac{r(x)}{h(x)}
    \]
    where $ax+b$, $a\neq0$ is the quotient and $r(x)$ is the remainder.
\end{itemize}

\section{Rational Functions 2}
\begin{enumerate}
    \item Collect all terms to the left side of the inequality and $0$ to the right side.
        \begin{eg}
            Solve $\frac{3}{x+1}<-\frac{2}{x-2}$:
            \[
                \frac{3}{x+1}+\frac{2}{x-2}>0
            \]
        \end{eg}
    \item Simplity the expression on the left side by finding a common denominator and adding the terms.
        \[
            \frac{3(x-2)}{(x+1)(x-2)}+\frac{2(x+1)}{(x+1)(x-2)}>0
        \]
    \item Once simplified, factor the numator, if possible, always leaving the denominator in factored form.
        \[
            \frac{5x-4}{(x+1)(x-2)}>0
        \]
    \item Create an internal table and identify the sign of each factor in the rational expression with each internal.
        \[
            \begin{array}{c|c|c|c|c}
                \multicolumn{1}{c}{} & \multicolumn{4}{l}{\hspace{2.45em} x=-1 \hspace{3em} x=\frac{4}{5} \hspace{2.85em} x=2} \\[1ex]
                \multicolumn{1}{c}{} & x<-1 & -1<x<\frac{4}{5} & \frac{4}{5}<x<2 & x>2 \\[0.5ex] \hline
                5x-4 & - & - & + & + \\ \hline
                x+1  & - & + & + & + \\ \hline
                x-2  & - & - & - & + \\ \hline\rule{0pt}{2.5ex}
                \frac{5x-4}{(x+1)(x+2)} & - & + & - & +
            \end{array}
        \]
        \begin{note}
            An extraneous solution is a solution of an equation derived from original equation that is not a solution of the original equation.
        \end{note}
\end{enumerate}