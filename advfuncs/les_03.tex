\chapter{Progressions}
\subsection{Arithmetic progression}
\begin{itemize}
\item $n$-th term: $a_n=a_1+(n-1)d$.
\item Sum: $S_n=\frac{n(a_1+a_n)}2\text{ or }na_1+\frac{n(n-1)}2d$.
\end{itemize}
\todo{Supplement attributes}
\subsection{Geometric progression}
\begin{itemize}
\item $n$-th term: $a_n=a_1q^{n-1}$.
\item Sum: $S_n=\frac{a_1-a_nq}{1-q}\text{ or }\frac{a_1(1-q^n)}{1-q}\quad(q\neq 1)$
\end{itemize}
\todo{Supplement attributes}
\chapter{Binomial Theorem}
\[
(x+y)^n=\binom{n}{k}x^ny^0+\binom{n}{1}x^{n-1}y^1+\binom{n}{2}x^{n-2}y^2+\cdots+\binom{n}{n}x^0y^n,
\]
where each $\binom{n}{k}$ is a positive integer known as a binomial coefficient, defined as
\[
\binom{n}{k}=\frac{n!}{k!(n-k)!}
\]
\begin{note}
The definition of combination is equal to the binomial coefficient
\[
C_k^n=\binom{n}{k}=\frac{n!}{k!(n-k)!},
\]
so the binomial theorem can also be written as
\[
(x+y)^n=C_n^0x^ny^0+C_n^1x^{n-1}y^1+C_n^2x^{n-2}y^2+\cdots+C_n^nx^0y^n
\]
\end{note}
