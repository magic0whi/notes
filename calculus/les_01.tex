%../preamble.tex
\chapter{Limits}
\section{Introduction to limits}
\begin{figure}[H]
    \centering
    \incfig{1limit_1.1}
    \caption{Left-/Right-hand limit}
\end{figure}
\begin{enumerate}
    \item Definition of limits

        Right-hand limit: $\lim_{x\to a^+} f(x)=R$, or $f(x)\to R\text{ as }x\to a^+$.

        Left-hand limit: $\lim_{x\to a^-} f(x)=L$, or $f(x)\to L\text{ as }x\to a^-$.

        Overall limit: If $\lim_{x\to a^+} f(x)=\lim_{x\to a^-} f(x)=L$, then $\lim_{x\to a} f(x)=L$

        Remember that $x$ is approaching $a$ but not equal $a$.
        \begin{definition}[Formal definition of limit]
            $\forall\epsilon>0$, $\exists\delta>0$ such that if $0<|x-a|<\delta$, then $|f(x)-L|<\epsilon$.

            That is, no matter how small $\epsilon$ gets, we can satisfy the condition $|f(x)-L|<\epsilon$ as long as $x$ get close enough to $a$; the proximity required is measured by $\delta$.
        \end{definition}
    \item Possible limit behaviors
        \begin{itemize}
            \item $\lim_{x\to a^+} f(x)=\lim_{x\to a^-}=\lim_{x\to a} f(x)$.
            \item $\lim_{x\to a^+} f(x)\neq\lim_{x\to a^-} f(x)$.
            \item Left-/Right-hand limit could fail to exist due to $\lim_{x\to a^+} f(x)=\pm\infty$ or/and $\lim_{x\to a^-} f(x)=\pm\infty$, because $\infty$ is not a real number.
            \item Left-/Right-hand limit could fail to exist because it oscillates between many values and never settle down.
        \end{itemize}
    \item The limit laws

        Addition: $\lim_{x\to a} [f(x)+g(x)]=L+M$.

        Subtraction: $\lim_{x\to a} [f(x)-g(x)]=L-M.$

        Multiplication: $\lim_{x\to a} [f(x)\cdot g(x)]=L\cdot M$.

        Division (Part 1): If $M\neq0$, then $\lim_{x\to a} \frac{f(x)}{g(x)}=\frac{L}{M}$.
\end{enumerate}

\section{Continuity}
\begin{enumerate}
    \item Definition of Continuity at A Point

        A function $f$ is \textbf{continuous at} $x=a$ if $\lim_{x\to a} f(x)=f(a)$.

        In particular, if either $f(a)$ or $\lim_{x\to a} f(x)$ fails to exist, then $f$ is discontinuous at $a$.

        We say that a function $f$ is \textbf{right-continuous} at $x=a$ if $\lim_{x\to a^+} f(x)=f(a)$.

        We say that a function $f$ is \textbf{left-continuous} at $x=a$ if $\lim_{x\to a^-} f(x)=f(a)$.
    \item Types of Discontinuities

        If $\lim_{x\to a^-} f(x)$ and $\lim_{x\to a^+} f(x)$ are exist but they are not equal, then we can say $f$ has a \textbf{jump discontinuity} at $x=a$.

        If $\lim_{x\to a} f(x)$ exists but does not equal $f(a)$, then we say that $f$ has a \textbf{removable discontinuity} at $x=a$.
        \begin{figure}[H]
            \centering
            \incfig{1limit_2.1}
            \caption{Jump discontinuity}
        \end{figure}
		\begin{figure}[H]
			\centering
			\incfig{1limit_2.2}
			\caption{Removable discontinuity}
		\end{figure}
   \item Definition of Continuous Functions

       A function $f(x)$ is \textbf{continuous} if for every point $c$ in he domain of $f(x)$, the function $f$ is continuous at the point $x=c$.
    \item Basic Continuous Functions

        The following functions are continuous at \textit{all real numbers}.
        \begin{itemize}
            \item all polynomials
            \item $\sqrt[3]{x}$ 
            \item $|x|$
            \item $\cos x$ and $\sin x$
            \item exponential functions $a^x$ with $a>0$.
        \end{itemize}

        The following functions are continuous \textit{at the specified values of $x$}.
        \begin{itemize}
            \item $\sqrt{x}$, for $x>0$.
            \item $\tan x$, at all $x$ where it is defined.
            \item logarithmic functions $\log_a x$ with base $a>0$, for $x>0$.
        \end{itemize}
    \item Limit Laws and Continuity

        If the functions $f$ and $g$ are continuous everywhere, then:
        \begin{itemize}
            \item $f\pm g$ is continuous everywhere.
            \item $f\cdot g$ is continuous everywhere.
            \item $\frac{f}{g}$ is continuous where it is defined.
            \item $f\circ g$ is continuous everywhere. ($\circ$ is the composition symbol which means $f\circ g(x)=f(g(x))$.)
        \end{itemize}
    \item Intermediate Value Theorem

        If $f$ is a function which is continuous on the interval $[a, b]$ and $M$ lies between the values of $f(a)$ and $f(b)$, then there is at least one point $c$ between $a$ and $b$ such that $f(c)=M$.

        (A function is \textbf{continuous on a closed interval} $[a, b]$ if it is continuous on interval $(a, b)$, right-continuous at $a$ and left-continuous at $b$.)
\end{enumerate}

\section{Limits of quotients}
If $\lim_{x\to a} f(x)=L$ and $\lim_{x\to a} g(x)=M$, then:
\begin{enumerate}
    \item If $M\neq 0$, then $\lim_{x\to a} \frac{f(x)}{g(x)}=\frac{L}{M}$
    \item If $M=0$ but $L\neq 0$, then $\lim_{x\to a} \frac{f(x)}{g(x)}$ does not exist.
    \item If both $M=0$ and $L=0$, then $\lim_{x\to a} \frac{f(x)}{g(x)}$ might exist, or it might not exist. More work is necessary to determine whether the last type of limit exists and what it is if it does exist.
\end{enumerate}

\chapter{The derivative}
\section{What is the derivative?}
\begin{enumerate}
    \item The definition of the average rate of change.

        The \textbf{average rate of change} of a function $f(x)$ over an interval $a\leq x\leq b$ is defined to be
        $$
        \frac{f(b)-f(a)}{b-a}.
        $$
    \item Definition of the derivative.

        The \textbf{derivative} of a function $f(x)$ at a point $x=a$ is defined to be
        $$
        f^\prime(a)=\lim_{b\to a}\frac{f(b)-f(a)}{b-a}.
        $$
\end{enumerate}

\section{Geometric Interpretation of The Derivative}
\begin{enumerate}
    \item \textbf{Secant lines}

        The \textit{secant line} of a function $f(x)$ over the interval $a\leq x\leq b$ is the line that passes through the points $(a,f(a))$ and $(b,f(b))$.
        \begin{itemize}
            \item The slope of the secant line is $\frac{f(b)-f(a)}{b-a}$, which is the average rate of change of the function $f(x)$ over the interval $a\leq x\leq b$.
            \item The equation of the secant line is $y=\frac{f(b)-f(a)}{b-a}(x-a)+f(a)$.
        \end{itemize}
    \item \textbf{Tangent lines}

        The \textit{tangent line} to a function $f(x)$ at the point $x=a$ is the line that passes through the point $(a,f(a))$, and whose slope is the instantaneous rate of change of $f(x)$ at the point $x=a$. This slope is the slope of the line you get if you imagine zooming in on the function until it looks like a line.
        \begin{itemize}
            \item The slope of the tangent line is $f^\prime(a)$.
            \item The equation for the tangent line is $y=f^\prime(a)(x-a)+f(a)$.
        \end{itemize}
        \paragraph{Properties of tangent lines.\\} If the derivative of $f(x)$ exists at $x=a$, then the tangent line exists. The tangent line may exist if the derivative is undefined at $x=a$ though. (Example $f(x)=\sqrt[3]{x}$ has a vertical tangent line at $x=0$.)
        \paragraph{What a tangent line is and is not\\} When introduced to tangent lines of circles, many students learn that a tangent is ``a line that touches the curve in only one point. '' This is true if your curve is a circle, but for many other curves and functions, this is \textbf{not a good} definition.
\begin{figure}[H]
    \centering
    \incfig{2derivative_2.1}
    \caption{The tangent line passes through multiple times of the function}
\end{figure}
\end{enumerate}

\section{The Derivative as a function}
\begin{enumerate}
    \item \textbf{Recall secant and tangent lines}

        In calculus, we often think of the function and its graph as being the same object. So to study functions, we often study their graphs.

        Recall: the derivative at a point is the slope of the tangent line to the graph through that point. But when our graph is nice and smooth, without any discontinuities, corners, or other weird behavior, we can find the slope of the tangent line at any point. Thus we can think of the derivative of a function as a function.
    \item \textbf{Example: The graph of a function and its derivative}
        \begin{figure}[H]
            \centering
            \incfig{2derivative_3.1}
            \caption{The graph of a function $y=g(x)$}
        \end{figure}
        \begin{figure}[H]
            \centering
            \incfig{2derivative_3.2}
            \caption{The graph of the derivative $y=g^\prime (x)$}
        \end{figure}
\end{enumerate}

\section{Calculating derivatives}
\begin{enumerate}
    \item \textbf{Derivatives of constant multiples}

        If $g(x)=kf(x)$ for some constant $k$, then
        $$
        g^\prime(x)=kf^\prime(x)
        $$
        at all points where $f$ is differentiable.
    \item \textbf{Derivatives of sums}

        If $h(x)=f(x)+g(x)$, then
        $$
        h^\prime(x)=f^\prime(x)+g^\prime(x)
        $$
        at all points where $f$ and $g$ are differentiable.
    \item \textbf{Derivatives of differences}

        Similarly, if $j(x)=f(x)-g(x)$, then
        $$
        j^\prime(x)=f^\prime(x)+g^\prime(x)
        $$
        at all points where $f$ and $g$ are differentiable.
    \item \textbf{Derivatives of constant multiples - proof}

        Suppose that $g(x)=kf(x)$ for all $x$, where $k$ is a constant. We want to prove that $g^\prime(x)=kf^\prime(x)$ at any point $x$ where $f$ is differentiable.

        We know that
        \begin{align*}
            g^\prime(x) & =\lim_{\Delta x\to 0} \frac{g(x+\Delta x)-g(x)}{\Delta x} \\
                        & =\lim_{\Delta x\to 0} \frac{kf(x+\Delta x)-kf(x)}{\Delta x} \\
                        & =\lim_{\Delta x\to 0} k\frac{f(x+\Delta x)-f(x)}{\Delta x} \\
                        & =\lim_{\Delta x\to 0} k \lim_{\Delta x\to 0} \frac{f(x+\Delta x)-f(x)}{\Delta x}
        \end{align*}
        The first limit is just $k$, and the second limit is the definition of $f^\prime(x)$. So we get $g^\prime=kf^\prime(x)$.
    \item \textbf{What is linearity?}

        We've seen that differentiation ``respects'' addition and multiplication by a constant. That is, if you take a derivative of a sum of functions, you get the same thing as if you differentiated each part, and then added the derivatives. Similarly, if you take the derivative of $k$ times a function, where $k$ is a constant, then you get $k$ times the derivative of the original function.

        Respecting addition and constant multiplication is this way is called ``linearity'', and it is an important property of the derivative operation!
    \item \textbf{The Power Rule}

        If $n$ is any fixed number, and $f(x)=x^n$, then $f^\prime(x)=nx^{n-1}$.
\end{enumerate}

\section{Leibniz notation}
\begin{enumerate}
    \item \textbf{Why Leibniz notation?}

        Why do we like to use Leibniz notation?

        The biggest reason is that it reminds us what the input variable is. The derivative is measuring the instantaneous rate of change of the output variable of a function with respect to the input variable. Sometimes, if there are lots of quantities that have variables representing them, it's easy to lose track of what is what; Leibniz notation helps to remind us.
    \item \textbf{Properties of Leibniz notation}
        \begin{itemize}
            \item \textbf{Units}: If $P$ has units of pressure, and $t$ has units of \textit{time}, then $\frac{\mathrm{d}P}{\mathrm{d}t}$ has units of pressure per time.
            \item \textbf{Evaluating at point}: If we want to take the derivative at a particular point $x=3$, then we use the notation $\left.\frac{\mathrm{d}f}{\mathrm{d}x}\right|_{x=3}$. The bar is read as ``evaluated at''.
            \item \textbf{Derivatives act on functions}
                
                We can write $\frac{\mathrm{d}(x^2)}{\mathrm{d}x}$ for the derivative of $x^2$.

                If a formula is long, we can write $\frac{\mathrm{d}}{\mathrm{d}y}(y^3+2y^2)$.
        \end{itemize}
\end{enumerate}

\section{Second derivatives and higher}
\begin{enumerate}
    \item \textbf{Second derivative}

        The second derivative of a function $f(x)$ is the first derivative of $f^\prime(x)$, and is denoted by $f^{\prime\prime}(x)$ or $\frac{\mathrm{d}^2 f}{\mathrm{d}x^2}$.
        \begin{proof}
            The second derivative in Leibinz notation is written by this transform
            $$
            f^{\prime\prime}(x)
            =\frac{\mathrm{d}}{\mathrm{d}x}\left(\frac{\mathrm{d}}{\mathrm{d}x}(f)\right)
            =\left(\frac{\mathrm{d}}{\mathrm{d}x}\right)^2(f)
            =\frac{\mathrm{d}^2 f}{\mathrm{d}x^2}
            $$
        \end{proof}
    \item \textbf{Higher derivatives}

        The $n$th derivative of a function $f(x)$ is the first derivative of $f^{(n-1)}(x)$, and is denoted by $f^{(n)}(x)$ or $\frac{\mathrm{d}^n f}{\mathrm{d} x^n}$.
    \item \textbf{Second derivative and concavity summary}
        
        On intervals where $f^{\prime\prime}>0$, the function $f$ is concave up.
        \begin{figure}[H]
            \centering
            \incfig{2derivative_6.1}
        \end{figure}

        On intervals where $f^{\prime\prime}<0$, the function $f$ is concave down.
        \begin{figure}[H]
            \centering
            \incfig{2derivative_6.2}
        \end{figure}

        Points where the graph of a function changes from concave up to concave down, or vice versa, are called \textbf{inflection points}.
    \item \textbf{Position, velocity, acceleration}

        If $x(t)$ is a function that describes position as a function of time, then:
        $x^\prime (t)$ is the velocity, and $x^{\prime\prime} (t)$ is the acceleration.
\end{enumerate}

\section{Trigonometric functions: sine and cosine}
\begin{enumerate}
    \item \textbf{Definition of significant figures}
        
        The number of \textit{significant figures} is the count of those digits that carry meaning with regards to precision.
        
        \textbf{Examples}
        \begin{itemize}
            \item All non-zero digits are significant \rule{20pt}{1pt} - 1235 has 4 significant digits.
            \item Zeros appearing between nonzero digits are significant - 101 has 3 significant digits.
            \item Trailing zeros in a number containing a decimal are significant - 32.000 has 5 significant figures.
        \end{itemize}
        \textbf{Non-examples}
        \begin{itemize}
            \item Trailing zeros in a number with no decimal are \textit{not} significant - 5400 has 2 significant figures.
            \item Leading zeros in decimal number are not significant - 0.0003 has 1 significant figure.
            \item Extraneous digits introduced in a computation to greater precision than measured data are \textit{not} significant - if .25 and .50 are measurements accurate to $\pm.01$, then in the product $(.25)(.50)=0.125$ the last 5 is \textit{not} significant.
        \end{itemize}
    \item \textbf{Derivative of sine and cosine}
        
        The derivative of the trig functions are:
        \begin{align*}
            & \frac{\mathrm{d}}{\mathrm{d}x}\sin(x)=\cos(x) \\
            & \frac{\mathrm{d}}{\mathrm{d}x}\cos(x)=-\sin(x) \\
            & \frac{\mathrm{d}^2}{\mathrm{d}x^2}\sin(x)=-\sin(x) \\
            & \frac{\mathrm{d}^2}{\mathrm{d}x^2}\cos(x)=-\cos(x)
        \end{align*}
        \begin{proof}
            To differentiate $\sin(x)$, first we using the trio sum formula
            \begin{align*}
                \frac{\sin(x+\mathrm{d}x)-\sin(x)}{\mathrm{d}x} & =\frac{\sin(x)\cos(\mathrm{d}x)+\cos(x)\sin(\mathrm{d}x)-\sin(x)}{\mathrm{d}x} & \\
                                                                & =\sin(x)\left(\frac{\cos(\mathrm{d}x)-1}{\mathrm{d}x}\right)+\cos(x)\left(\frac{\sin(\mathrm{d}x)}{\mathrm{d}x}\right) \\
                                                                & =\sin(x)\cos(0)^\prime+\cos(x)\sin(0)^\prime
            \end{align*}
            To proof $\cos(0)^\prime=0$ and $\sin(0)^\prime=1$ using geometry, first we draw a unit circle. For convenience, replace $x$ with $\theta$.
            \begin{figure}[H]
                \centering
                \incfig{2derivative_7.1}
                \caption{Unit circle with angle $\theta$ (Radian)}
            \end{figure}
            To zoom in and make angle $\theta$ smaller, we found
            \begin{figure}[H]
                \centering
                \incfig{2derivative_7.2}
                \caption{Zoomed in $\theta$ (Radian)}
            \end{figure}
            while $\theta\to 0$, $1-\cos(\theta)$ getting smaller \textbf{faster} than $\theta$ itself, and $\sin(\theta)$ approaching the green arc $\theta$. Therefore $\frac{\cos(\mathrm{d}\theta)-1}{\mathrm{d}\theta}=0$, $\frac{\sin(\mathrm{d}\theta)}{\mathrm{d}\theta}=1$. Which is $\cos(0)^\prime=0$ and $\sin(0)^\prime=1$

            Summarize, $\sin^\prime(x)=\sin(x)\cos^\prime(0)+\cos(x)\sin^\prime(0)=\sin(x)\cdot 0+\cos(x)\cdot 1=\cos(x)$

            So as to cosine, differentiate $\cos(x)$ we get $\cos^\prime(x)=\cos(x)\frac{\cos(\mathrm{d}x)}{\mathrm{d}x}-\sin(x)\frac{\sin(\mathrm{d}x)}{\mathrm{d}}=\cos(x)\cos^\prime(0)-\sin(x)\sin^\prime(0)=-\sin(x)$
            \end{proof}
\end{enumerate}
