%../preamble.tex
\chapter{Limits}
\section{Introduction to limits}
\begin{figure}[H]
    \centering
    \incfig{1limit_1.1}
    \caption{Left-/Right-hand limit}
\end{figure}
\begin{enumerate}
    \item Definition of limits \\
        Right-hand limit: $\lim_{x\to a^+} f(x)=R$, or $f(x)\to R\text{ as }x\to a^+$. \\
        Left-hand limit: $\lim_{x\to a^-} f(x)=L$, or $f(x)\to L\text{ as }x\to a^-$. \\
        Overall limit: If $\lim_{x\to a^+} f(x)=\lim_{x\to a^-} f(x)=L$, then $\lim_{x\to a} f(x)=L$ \\
        Remember that $x$ is approaching $a$ but not equal $a$.
        \begin{definition}[Formal definition of limit]
            $\forall\epsilon>0$, $\exists\delta>0$ such that if $0<|x-a|<\delta$, then $|f(x)-L|<\epsilon$.

            That is, no matter how small $\epsilon$ gets, we can satisfy the condition $|f(x)-L|<\epsilon$ as long as $x$ get close enough to $a$; the proximity required is measured by $\delta$.
        \end{definition}
    \item Possible limit behaviors \\
        \begin{itemize}
            \item $\lim_{x\to a^+} f(x)=\lim_{x\to a^-}=\lim_{x\to a} f(x)$.
            \item $\lim_{x\to a^+} f(x)\neq\lim_{x\to a^-} f(x)$.
            \item Left-/Right-hand limit could fail to exist due to $\lim_{x\to a^+} f(x)=\pm\infty$ or/and $\lim_{x\to a^-} f(x)=\pm\infty$, because $\infty$ is not a real number.
            \item Left-/Right-hand limit could fail to exist because it oscillates between many values and never settle down.
        \end{itemize}
    \item The limit laws \\
    Addition: $\lim_{x\to a} [f(x)+g(x)]=L+M$. \\
    Subtraction: $\lim_{x\to a} [f(x)-g(x)]=L-M.$ \\
    Multiplication: $\lim_{x\to a} [f(x)\cdot g(x)]=L\cdot M$. \\
    Division (Part 1): If $M\neq0$, then $\lim_{x\to a} \frac{f(x)}{g(x)}=\frac{L}{M}$.
\end{enumerate}

\section{Continuity}
\begin{enumerate}
    \item Definition of Continuity at A Point \\
        A function $f$ is \textbf{continuous at} $x=a$ if $\lim_{x\to a} f(x)=f(a)$.

        In particular, if either $f(a)$ or $\lim_{x\to a} f(x)$ fails to exist, then $f$ is discontinuous at $a$. \\
        We say that a function $f$ is \textbf{right-continuous} at $x=a$ if $\lim_{x\to a^+} f(x)=f(a)$. \\
        We say that a function $f$ is \textbf{left-continuous} at $x=a$ if $\lim_{x\to a^-} f(x)=f(a)$.
    \item Types of Discontinuities \\
        If $\lim_{x\to a^-} f(x)$ and $\lim_{x\to a^+} f(x)$ are exist but they are not equal, then we can say $f$ has a \textbf{jump discontinuity} at $x=a$.
        \begin{figure}[H]
            \centering
            \incfig{1limit_2.1}
            \caption{Jump discontinuity}
        \end{figure}
        If $\lim_{x\to a} f(x)$ exists but does not equal $f(a)$, then we say that $f$ has a \textbf{removable discontinuity} at $x=a$.
		\begin{figure}[H]
			\centering
			\incfig{1limit_2.2}
			\caption{Removable discontinuity}
		\end{figure}
   \item Definition of Continuous Functions \\
       A function $f(x)$ is \textbf{continuous} if for every point $c$ in he domain of $f(x)$, the function $f$ is continuous at the point $x=c$.
    \item Basic Continuous Functions \\
        The following functions are continuous at \textit{all real numbers}.
        \begin{itemize}
            \item all polynomials
            \item $\sqrt[3]{x}$ 
            \item $|x|$
            \item $\cos x$ and $\sin x$
            \item exponential functions $a^x$ with $a>0$.
        \end{itemize}
        The following functions are continuous \textit{at the specified values of $x$}.
        \begin{itemize}
            \item $\sqrt{x}$, for $x>0$.
            \item $\tan x$, at all $x$ where it is defined.
            \item logarithmic functions $\log_a x$ with base $a>0$, for $x>0$.
        \end{itemize}
    \item Limit Laws and Continuity \\
        If the functions $f$ and $g$ are continuous everywhere, then:
        \begin{itemize}
            \item $f\pm g$ is continuous everywhere.
            \item $f\cdot g$ is continuous everywhere.
            \item $\frac{f}{g}$ is continuous where it is defined.
            \item $f\circ g$ is continuous everywhere. ($\circ$ is the composition symbol which means $f\circ g(x)=f(g(x))$.)
        \end{itemize}
    \item Intermediate Value Theorem \\
        If $f$ is a function which is continuous on the interval $[a, b]$ and $M$ lies between the values of $f(a)$ and $f(b)$, then there is at least one point $c$ between $a$ and $b$ such that $f(c)=M$. \\
        (A function is \textbf{continuous on a closed interval} $[a, b]$ if it is continuous on interval $(a, b)$, right-continuous at $a$, and left-continuous at $b$.)
\end{enumerate}
