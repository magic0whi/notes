\chapter{Limits}
\section{Introduction to Limits}
\begin{figure}[H]
  \centering
  \incfig{1limit_1.1}
  \caption{Left/Right-hand limit}
\end{figure}

\subsection{Definition of Limits}
\begin{itemize}
\item If either $f(a)$ or $\lim_{x\to a} f(x)$ fails to exist, then $f$ is discontinuous at $a$.
\item Right-hand limit: $\lim_{x\to a^+}f(x)=R$, or $f(x)\to R$ as $x\to a^+$;
\item Left-hand limit: $\lim_{x\to a^-}f(x)=L$, or $f(x)\to L$ as $x\to a^-$;
\item Overall limit: If $\lim_{x\to a^+}f(x)=\lim_{x\to a^-}f(x)=L$, then $\lim_{x\to a}f(x)=L$.
\end{itemize}

\begin{note}Remember that $x$ is approaching $a$ but not necessary equals $a$.\end{note}
\begin{definition}[Formal definition of limit]
\[
\forall\epsilon>0,\ \exists\delta>0\ \text{that if}\ 0<\lvert x-a\rvert<\delta,\ \text{then}\ \lvert f(x)-L\rvert<\epsilon.
\]
That is, no matter how small $\epsilon$ gets, we can satisfy the condition $\lvert f(x)-L\rvert<\epsilon$ as long as $x$ get close enough to $a$. The proximity required is measured by $\delta$.
\end{definition}

\subsection{Possible Limit Behaviors}
\begin{itemize}
\item $\lim_{x\to a^+}f(x)=\lim_{x\to a^-}f(x)=\lim_{x\to a}f(x)$;
\item $\lim_{x\to a^+}f(x)\neq\lim_{x\to a^-}f(x)$;
\item Left\slash Right-hand limit could fail to exist due to $\lim_{x\to a^+}f(x)=\pm\infty$ or $\lim_{x\to a^-}f(x)=\pm\infty$, because $\infty$ is not a real number;
\item Left\slash Right-hand limit could fail to exist because it oscillates between many values and never settle down.
\end{itemize}

\subsection{The Limit Laws}
\begin{itemize}
\item Addition: $\lim_{x\to a}[f(x)+g(x)]=L+M$;
\item Subtraction: $\lim_{x\to a}[f(x)-g(x)]=L-M$;
\item Multiplication: $\lim_{x\to a}[f(x)\cdot g(x)]=L\cdot M$;
\item Division (Part 1): If $M\neq 0$, then $\lim_{x\to a}f(x)/g(x)=L/M$.
\end{itemize}


\section{Continuity}
\subsection{Definition of Continuity at a Point}
\begin{itemize}
\item A function $f$ is \textbf{continuous at} $x=a$ if $\lim_{x\to a}f(x)=f(a)$;
\item If either $f(a)$ or $\lim_{x\to a} f(x)$ fails to exist, then $f$ is discontinuous at $a$;
\item We say that a function $f$ is \textit{right-continuous} at $x=a$ if $\lim_{x\to a^+}f(x)=f(a)$;
\item We say that a function $f$ is \textit{left-continuous} at $x=a$ if $\lim_{x\to a^-}f(x)=f(a)$.
\end{itemize}

\subsection{Types of Discontinuities}
\begin{itemize}
  \item If $\lim_{x\to a^-}f(x)$ and $\lim_{x\to a^+}f(x)$ exist but not equal, then we can say $f$ has a \textbf{jump discontinuity} at $x=a$.
  \item If $\lim_{x\to a}f(x)$ exists but doesn't equal $f(a)$, then we say that $f$ has a \textbf{removable discontinuity} at $x=a$.
\end{itemize}
\begin{figure}[H]\centering\incfig{1limit_2.1}\caption{Jump discontinuity}\end{figure}
\begin{figure}[H]\centering\incfig{1limit_2.2}\caption{Removable discontinuity}\end{figure}

\subsection{Continuity of Basic Continuous Functions}
\begin{definition}[Definition of continuous functions]
  A function $f(x)$ is \textbf{continuous} if for every point $c$ in the domain of $f(x)$, the function $f$ is continuous at the point $x=c$.
\end{definition}
The following functions are continuous at $x\in\R$.
\begin{itemize}
\item All polynomials;
\item $\sqrt[3]{x}$;
\item $\abs x$;
\item $\cos x$, $\sin x$;
\item $a^x\ (a>0)$;
\item etc.
\end{itemize}

The following functions are continuous \textit{at the specified values of $x$}.
\begin{itemize}
\item $\sqrt x\ (x>0)$;
\item $\tan x$, at all $x$ where it's defined;
\item $\log_a x\ (a>0,\ x>0)$;
\item etc.
\end{itemize}

\subsection{Limit Laws and Continuity}
If the functions $f$ and $g$ are continuous everywhere, then:
\begin{itemize}
\item $f\pm g$ is continuous everywhere;
\item $f\cdot g$ is continuous everywhere;
\item $f/g$ is continuous where it's defined.
\item $f\circ g$ is continuous everywhere. (``$\circ$'' is the composition symbol which means $f\circ g(x)=f[g(x)]$)
\end{itemize}

\subsection{Intermediate Value Theorem}
If $f$ is a function which is continuous on the interval $\left[a,b\right]$ and $M$ lies between the values of \(f(a)\) and \(f(b)\), then there is at least one point $c$ between $a$ and $b$ such that $f(c)=M$.

\begin{definition}
  A function is \textbf{continuous on a closed interval} $\left[a,b\right]$ if it is continuous on interval $\left(a,b\right)$, right-continuous at $a$ and left-continuous at $b$.
\end{definition}
\begin{note}IVT is intimately tied in with the properties of the real number system.\end{note}


\section{Limits of Quotients}
If $\lim_{x\to a}f(x)=T$ and $\lim_{x\to a}g(x)=B$, then:
\begin{itemize}
\item If $T\neq0$, then $\lim_{x\to a}f(x)/g(x)=T/B$;
\item If $B=0$ but $T\neq0$, then $\lim_{x\to a}f(x)/g(x)$ does not exist;
\item If both $T=B=0$, then $\lim_{x\to a}f(x)/g(x)$ might exist, or it might not exist. More work is necessary to determine whether the last type of limit exists and what it is if it does exist.
\end{itemize}



\chapter{The Derivative}
\section{What is the Derivative?}
\subsection{Average Rate of Change}
The \textit{average rate of change} of a function $f(x)$ over $x\in\left[a,b\right]$ is defined to be
\[\frac{f(b)-f(a)}{b-a}.\]

\subsection{Definition of the Derivative}
The \textit{derivative} of a function $f(x)$ at a point $x=a$ is defined to be
\[f^\prime(a)=\lim_{b\to a}\frac{f(b)-f(a)}{b-a}.\]


\section{Geometric Interpretation of the Derivative}
\subsection{Secant Lines}
The \textit{secant line} of a function $f(x)$ over $x\in\left[a,b\right]$ is the line that passes through the points $(a,f(a))$ and $(b,f(b))$.
\begin{itemize}
  \item The slope of the secant line is $\frac{f(b)-f(a)}{b-a}$, which is the average rate of change of the function $f(x)$ over the interval $a\leq x\leq b$.
  \item The equation of the secant line is $y=\frac{f(b)-f(a)}{b-a}(x-a)+f(a)$.
\end{itemize}

\subsection{Tangent Lines}
The \textit{tangent line} to a function $f(x)$ at the point $x=a$ is the line that passes through the point $(a,f(a))$, and whose slope is the instantaneous rate of change of $f(x)$ at the point $x=a$.
\begin{remark}
This slope is of the line you get if you imagine zooming in on the function until it looks like a line.
\end{remark}
\begin{itemize}
  \item The slope of the tangent line is $f^\prime(a)$.
  \item The equation of the tangent line is $y=f^\prime(a)(x-a)+f(a)$.
\end{itemize}

\subsection{Properties of Tangent Lines}
If $f^\prime(a)$ exists, then the tangent line exists as well. Though, the tangent line may exist even the derivative is undefined at the point $x=a$ (e.g. $f(x)=\sqrt[3]x$ has a vertical tangent line at $x=0$).

\subsection{What a Tangent Line Is and Isn't}
When introduced to tangent lines of circles, we learn that a tangent is ``a line that touches the curve in only one point.'' This is true if your curve is a circle, but for many other curves and functions, this definition is ill-formed.
\begin{figure}[H]
  \centering
  \incfig{2derivative_2.1}
  \caption{The tangent line passes through multiple times of the function}
\end{figure}


\section{The Derivative as a Function}
The derivative at a point is the slope of the tangent line to the graph through that point. But when our graph is nice and smooth, without any discontinuities, corners, or other weird behavior, we can find the slope of the tangent line at any point. Thus we can think of the derivative of a function as a function.
\begin{figure}[H]
    \centering
    \incfig{2derivative_3.1}
    \caption{The graph of a function $y=g(x)$}
\end{figure}
\begin{figure}[H]
    \centering
    \incfig{2derivative_3.2}
    \caption{The graph of the derivative $y=g^\prime(x)$}
\end{figure}


\section{The Linearity of Derivatives}
\subsection{Derivatives of Constant Multiples}
For functions $g$ and $f$,
\[\text{if}\ g=kf\ \text{for some constant}\ k,\ \text{then}\ g^\prime=kf^\prime\]
at all points where $f$ is differentiable.
\begin{definition}[Differentiation]
The process of finding a derivative is called differentiation.
\end{definition}
\begin{proof}
  Suppose that $g(x)=kf(x)$ for all $x$, where $k$ is a constant. We want to prove that $g^\prime(x)=kf^\prime(x)$ at any point $x$ where $f$ is differentiable.\par
  We know that
  \begin{align*}
    g^\prime(x) & =\lim_{\Delta x\to 0}\frac{g(x+\Delta x)-g(x)}{\Delta x}\\
                & =\lim_{\Delta x\to 0}\frac{kf(x+\Delta x)-kf(x)}{\Delta x}\\
                & =\lim_{\Delta x\to 0}k\frac{f(x+\Delta x)-f(x)}{\Delta x}\\
                & =\lim_{\Delta x\to 0}k\lim_{\Delta x\to 0}\frac{f(x+\Delta x)-f(x)}{\Delta x}
  \end{align*}
  The first limit is just $k$, and the second limit is the definition of $f^\prime$. So we get $g^\prime=kf^\prime$.
\end{proof}

\subsection{Derivatives of Sums/Differences}
For functions $f$, $g$, $h$,
\[\text{if}\ h=f\pm g,\ \text{then}\ h^\prime=f^\prime\pm g^\prime\]
at all points where $f$ and $g$ are differentiable.

\subsection{What is Linearity?}
We've seen that derivatives ``respects'' addition and multiplication by a constant. Respecting addition and constant multiplication in this way is called \textit{linearity}.

\subsection{The Power Rule}
If $n$ is any fixed number, and $f(x)=x^n$, then $f^\prime(x)=nx^{n-1}$.

\section{Leibniz Notation}
\subsection{Why do we use Leibniz notation?}
The biggest reason is that it reminds us what the independent variable is. The derivative is measuring the instantaneous rate of change of the implicit variable of a function respects to the independent variable. Sometimes, if there are lots of quantities that have variables representing them, it's easy to lose track of what is what. Leibniz notation helps to remind us.

\subsection{Definition of Leibniz Notation}
For function $f(x)$ and its derivative funtion $f^\prime(x)$, has
\[f^\prime(x)=\frac{\dd f(x)}{\dd x}\]
Where
\begin{gather*}
\mathrm{d}f(x)=f(x+\mathrm{d}x)-f(x)\quad\equiv\quad\lim_{\Delta x\to0}\Delta f(x)=\lim_{\Delta x\to0}[f(x+\Delta x)-f(x)],\\
\end{gather*}
\begin{corollary}[$\dd x=\lim_{\Delta x\to0}\Delta x$]
  Substitute $f(x)=x$ gets
  \begin{align*}
    \dd x=(x+\dd x)-x\equiv\lim_{\Delta x\to0}\Delta x=\lim_{\Delta x\to0}(x+\Delta x)-x\\
    \therefore\mathrm{d}x=\lim_{\Delta x\to0}\Delta x.
  \end{align*}
\end{corollary}
\subsection{Properties of Leibniz Notation}
\begin{itemize}
\item Units: If $P$ has units of pressure, and $t$ has units of \textit{time}, then $\dd P/\dd t$ has units of pressure per time.
\item Evaluating at point: If we want to take the derivative at a particular point $x=3$, then we use the notation $\evalat{\frac{\dd f}{\dd x}}{x=3}$. The bar is read as ``evaluated at''.
\end{itemize}

\subsection{Derivatives Act on Functions}
We can write $\dd x^2/\dd x$ for the derivative of $x^2$. If a formula is long, we can write $\frac\dd{\dd y}(y^3+2y^2)$.


\section{The Second Derivatives and Higher}
\subsection{The Second Derivative}
The second derivative of a function $f$ is the first derivative of $f^\prime$, and is denoted by $f^{\prime\prime}$ or $\dd^2 f/\dd x^2$.
\begin{proof}
  The second derivative in Leibniz notation is written by this transform
  \[
  f^{\prime\prime}
  =\frac\dd{\dd x}\left(\frac\dd{\dd x}f\right)
  =\left(\frac\dd{\dd x}\right)^2 f
  =\frac{\dd^2 f}{\dd x^2}
  \]
\end{proof}

\subsection{Higher Derivatives}
The $n$-th derivative of a function $f$ is the first derivative of $f^{(n-1)}$, and is denoted by $f^{(n)}$ or $\dd^n f/\dd x^n$.

\subsection{Functions's Concavity with Second Derivative}
\begin{itemize}
\item On intervals where $f^{\prime\prime}>0$, the function $f$ is concave up.
\item On intervals where $f^{\prime\prime}<0$, the function $f$ is concave down.
\end{itemize}
\begin{figure}[H]
  \centering
  \incfig{2derivative_6.1}
  \caption{$f^\prime$ bigger and bigger, $f^{\prime\prime}>0$, concave up}
\end{figure}
\begin{figure}[H]
  \centering
  \incfig{2derivative_6.2}
  \caption{$f^\prime$ smaller and smaller, $f^{\prime\prime}<0$, concave down}
\end{figure}
\begin{definition}[Inflection points]
Points where the graph of a function changes from concave up to concave down, or vice versa, are called \textit{inflection points}.
\end{definition}

\subsection{Physical Sense of Derivatives: Position, Velocity, Acceleration}
If $p(t)$ is a function that describes position at time $t$, then $p^\prime(t)$ is the velocity, and $p^{\prime\prime}(t)$ is the acceleration.

\section{Derivatives of Trigonometric Functions: Sine and Cosine}
The derivative of the trig functions are:
\begin{align*}
  \frac\dd{\dd x}\sin x=       & \mathbin{\phantom-}\cos x\\
  \frac\dd{\dd x}\cos x=       & -\sin x\\
  \frac{\dd^2}{\dd x^2}\sin x= & -\sin x\\
  \frac{\dd^2}{\dd x^2}\cos x= & -\cos x
\end{align*}
\begin{proof}
To differentiate $\sin x$, first use the trig sum formula
\begin{align*}
  \frac{\sin(x+\dd x)-\sin x}{\dd x} & =\frac{\sin x\cos\dd x+\cos x\sin\dd x-\sin x}{\dd x} & \\
  & =\sin x\left(\frac{\cos\dd x-1}{\dd x}\right)+\cos x\left(\frac{\sin\dd x}{\dd x}\right)\\
  & =\sin x\cos^\prime0+\cos x\sin^\prime0
\end{align*}

To proof $\cos^\prime0=0$ and $\sin^\prime0=1$, I'll show it in geometry, first draw a unit circle. For convenience, replace $x$ with $\theta$ (Figure \ref{fig:1}). Then zoom in and make the angle $\theta$ smaller (Figure \ref{fig:2}), we'll found while $\theta\to0$, $1-\cos\theta$ getting smaller faster than $\theta$ itself, and \(\sin\theta\) is approaching the arc $\theta$. Therefore $(\cos\dd\theta-1)/\dd\theta=0$ and $\sin\dd\theta/\dd\theta=1$. Which is $\cos^\prime0=0$ and $\sin^\prime0=1$.

Therefore, $\sin^\prime x=\cos x$.

Similarily, to cosine, differentiate $\cos x$ got $\cos^\prime x=\cos x\left(\frac{\cos\dd x-1}{\dd x}\right)-\sin x\left(\frac{\sin\dd x}{\dd x}\right)=\cos x\cos^\prime 0-\sin x\sin^\prime 0=-\sin x$.
\end{proof}
\begin{figure}[H]
  \centering
  \incfig{2derivative_7.1}
  \caption{Unit circle with angle $\theta$}
  \label{fig:1}
\end{figure}
\begin{figure}[H]
  \centering
  \incfig{2derivative_7.2}
  \caption{Zoomed in $\theta$}
  \label{fig:2}
\end{figure}

\subsection{Definition of Significant Figures}
\todo{Move this section to appendix}
The number of \textit{significant figures} is the count of those digits that carry meaning with regards to precision.
\begin{eg}\leavevmode
\begin{itemize}
  \item All non-zero digits are significant: 1235 has 4 significant digits.
  \item Zeros appearing between nonzero digits are significant: 101 has 3 significant digits.
  \item Trailing zeros in a number containing a decimal are significant: 32.000 has 5 significant figures.
\end{itemize}
\textsf{\textbf{\color{NavyBlue!70!Black}Non-examples.}}
\begin{itemize}
  \item Trailing zeros in a number with no decimal are \textbf{not} significant: 5400 has 2 significant figures.
  \item Leading zeros in decimal number are \textbf{not} significant: 0.0003 has 1 significant figure.
  \item Extraneous digits introduced in a computation to greater precision than measured data are \textit{not} significant: if .25 and .50 are measurements accurate to $\pm.01$, then in the product $(.25)(.50)=0.125$ the last 5 is \textbf{not} significant.
\end{itemize}
\end{eg}
