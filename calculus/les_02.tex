\chapter{Differentiation}
\section{Linear Approximation}
The linear approximation for a function $f$ near point $x=a$ is given by the following equivalent formulas:
\begin{gather*}
  \Delta f\approx\evalat{\frac{\dd f}{\dd x}}{x=a}\cdot\Delta x\ \text{for}\ \Delta x\ \text{near}\ 0,\\
  f(x)\approx f^\prime(a)(x-a)+f(a)\ \text{for}\ x\ \text{near}\ a.
\end{gather*}

\section{The Product Rule}
\[\text{If}\ h=f\cdot g,\ \text{then}\ h^\prime=f\cdot g^\prime+f^\prime\cdot g.\]
\begin{proof}
  To proof the product rule, we can write $f(x+\dd x)=\dd f(x)+f(x)$, the reason why this equation is important is that if we differentiate the $h=fg$:
  \begin{align*}
    \frac{\dd h}{\dd x} & =\frac\dd{\dd x}(fg)\\
                        & =\frac{f(x+\dd x)g(x+\dd x)-f(x)g(x)}{\mathrm{d}x}\\
                        & =\frac{(\dd f+f)(\dd g+g)-fg}{\dd x}\\
                        & =\frac{\dd f\dd g+g\dd f+f\dd g+\cancel{fg-fg}}{\dd x}\\
                        & =\frac{\dd f\dd g}{\dd x}+g\frac{\dd f}{\dd x}+f\frac{\dd g}{\dd x}.
  \end{align*}
  The $\frac{\dd f\dd g}{\dd x}\) can be transformed into $\frac{\dd f}{\dd x}\cdot\frac{\dd g}{\dd x}\cdot\dd x$ which would be 0 because the extra $\dd x$ at the end. Finally
  \[
    \frac{\dd h}{\dd x}=g\frac{\dd f}{\dd x}+f\frac{\dd g}{\dd x}.
  \]
\end{proof}

\section{The Quotient Rule}
\subsection{The Quotient Rule}
For functions $f$ and $g$,
\[
\text{if}\ h=\frac f g\ ,\ \text{then}\ h^\prime=\frac{f^\prime g-fg^\prime}{g^2}
\]
at all points where $f$ and $g$ are differentiable and $g\neq0$.

\subsection{Derivatives of All Trigonometric Functions}
\begin{align*}
  & \frac\dd{\dd x}\sin x=\cos x\\
  & \frac\dd{\dd x}\cos x=-\sin x\\
  & \frac\dd{\dd x}\tan x=\cos^{-2}x=\sec^2 x\\
  & \frac\dd{\dd x}\cot x=-\sin^{-2}x=-\csc^2 x\\
  & \frac\dd{\dd x}\sec x=\frac{\sin x}{\cos^2 x}=\sec x\tan x\\
  & \frac\dd{\dd x}\csc x=-\frac{\cos x}{\sin^2 x}=-\csc x\cot x.
\end{align*}

\section{The Chain Rule}
For functions $f$ and $g$,
\[\text{if}\ h=f\circ g,\ \text{then}\ h^\prime=f^\prime(g)g^\prime\]
at all points where $f^\prime(g)$ and $g^\prime$ are defined.
\begin{note}Notice that $f^\prime(g)\neq[f(g)]^\prime $.\end{note}
Alternatively, let $y=f(u)$ and $u=g(x)$, then
\[
  \evalat{\frac{\dd y}{\dd x}}{x=a}=\evalat{\frac{\dd y}{\dd u}}{u=g(a)}\cdot\evalat{\frac{\dd u}{\dd x}}{x=a}
\]
at any point $x=a$ where the derivatives on the right side are defined.
\todo{Here}
\begin{proof}[Quotient Rule (by using Chain Rule and Product Rule)]
If \(h(x)=\frac{f(x)}{g(x)}\), the first thing to do is to rewrite \(\frac{1}{g(x)}\) in terms of a fractional power \(h(x)=f(x)\cdot g^{-1}(x)\). To help keep clear we write \(u=g(x)\) so \(h(x)=f(x)\cdot u^{-1}\). Thus the last step is a product, then we start by product rule
    \begin{align*}
        h^\prime (x) & =f^\prime (x)\cdot u^{-1}+(u^{-1})^\prime f(x)\\
                     & =\frac{f^\prime (x)}{g(x)}+(-u^{-2}\cdot u^\prime)f(x)\\
                     & =\frac{f^\prime (x)}{u}+[-\frac{1}{g^2(x)}\cdot g^\prime (x)]f(x)\\
                     & =\frac{f^\prime (x)}{g(x)}-\frac{g^\prime (x)f(x)}{g^2(x)}\\
                     & =\frac{f^\prime (x)g(x)+g^\prime (x)f(x)}{g^2(x)}.
    \end{align*}
\end{proof}

\section{Implicit Differentiation}
\begin{note}\textbf{Implicit functions:} An implicit function is an equation involving both \(x\) and \(y\) (or any two variables really); you \textit{could} solve for \(y\) as a function of \(x\), but often times that computation is messy or impossible. If a function is not implicit, we say that it is explicit.
\end{note}

Immediately differentiating \(y=x^{m/n}\) is hard. But differentiating the \textit{implicit} function \(y^n=x^m\) was a whole lot easier. This happens any time your function is more simply described implicitly.
\begin{align*}
    \phantom{\Leftrightarrow\,} & \frac{\mathrm{d}}{\mathrm{d}x}(y^n=x^m)\\
                             \Leftrightarrow\, & ny^{n-1}\cdot\frac{\mathrm{d}y}{\mathrm{d}x}=mx^{m-1}\\
                             \Leftrightarrow\, & \frac{\mathrm{d}y}{\mathrm{d}x}=\frac{m}{n}\cdot\frac{x^{m-1}}{y^{n-1}}.
\end{align*}

\begin{eg}
    Which of the following equations give \(y\) as an implicit function of \(x\)?
    \begin{itemize}
        \item \(y^3=x^2\) is implicit. 
        \item \(y=x^{2/3}\) is explicit.
        \item \(yx=1\) is implicit.
        \item \(y+x=1\) is implicit.
        \item \(\sin(y)=\cos(x)\) is implicit.
    \end{itemize}
\end{eg}

\section{Inverse Function}
\begin{enumerate}
    \item \textbf{Definition of inverse function}\\
        \indent If function \(f\) and \(g\) satisfy \(g[f(x)]=x\) and \(f[g(y)]=y\), then we say \(g\) is the inverse of \(f\), and denote it by \(f^{-1}\). (Similarly, \(f=g^{-1}\).)\\
        \indent If a function \(f\) has an inverse function \(f^{-1}\), then \(f^{-1}(b)=a\) if and only if \(b=f(a)\).
    \item \textbf{Definition of one-to-one}\\
        \indent A function \(f\) is \textbf{one-to-one} if whenever \(a\neq b\), \(f(a)\neq f(b)\). (Horizontal line test.)
    \item \textbf{Domain and range, interval notation}\\
        \indent The \textbf{domain} of a function \(f\) is the set of allowable input values.\\
        \indent For instance, the domain of the function \(f(x)=1/x\) is the set of all non-zero real numbers. (\(x\in(-\infty,0)\cup(0,+\infty)\).)\par
        The \textbf{range} of \(f\) is the set of all possible output values. For instance, the range of the function \(g(x)=x^2\) is the set of all real numbers that are non-negative. (\(g(x)\in[0,+\infty)\).)\par
        An \textbf{open interval}, denoted \((a,b)\), is the set of numbers \(x\) such that \(a<x<b\). This notation using round parentheses for open intervals is not universal. Many mathematicians use reversed square brackets instead. For instance, they would denote the interval \(3<x<7\) as \(]3,7[\) rather than \((3,7)\). However, I will stick to round parentheses for open intervals.
    \item \textbf{The inverse trig functions}
        \begin{align*}
            & \sin\theta=x\text{ such that }\arcsin x=\theta, \theta\in[-\frac{\pi}{2},\frac{\pi}{2}]\\
            & \cos\theta=x\text{ such that }\arccos x=\theta, \theta\in[0,\pi]\\
            & \tan\theta=x\text{ such that }\arctan x=\theta, \theta\in(-\frac{\pi}{2},\frac{\pi}{2}).
        \end{align*}
    \item \textbf{Derivatives of inverse functions}\\
        \indent If \(g\) is a (full or partial) inverse of a function \(f\), then
        \[
        g^\prime (x)=\frac{1}{f^\prime [g(x)]}.
        \]
        at all \(x\) where \(f^\prime [g(x)]\) exists and is non-zero.
    \item \textbf{Derivatives of arctrig functions}\\
        \indent Where defined,
        \begin{align*}
            & \frac{\mathrm{d}}{\mathrm{d}x}\arcsin x=\frac{1}{\sqrt{1-x^2}}\\
            & \frac{\mathrm{d}}{\mathrm{d}x}\arccos x=-\frac{1}{\sqrt{1-x^2}}\\
            & \frac{\mathrm{d}}{\mathrm{d}x}\arctan x=\frac{1}{x^2+1}.
        \end{align*}
        \begin{proof}\leavevmode
            \begin{itemize}
                \item Assume \(\arcsin x=\theta, \theta\in[-\frac{\pi}{2},\frac{\pi}{2}]\),
                    \[
                        \frac{\mathrm{d}}{\mathrm{d}x}\arcsin x=\frac{1}{\cos\theta}=\frac{1}{\sqrt{1-\sin^2 \theta}}=\frac{1}{\sqrt{1-x^2}}.
                    \]
                \item Assume \(\arccos x=\theta, \theta\in[0,\pi]\),
                    \[
                        \frac{\mathrm{d}}{\mathrm{d}x}\arccos x=\frac{1}{-\sin\theta}=\frac{1}{-\sqrt{1-\cos^2 \theta}}=-\frac{1}{\sqrt{1-x^2}}.
                    \]
                \item Assume \(\arctan x=\theta, \theta\in(-\frac{\pi}{2},\frac{\pi}{2})\),
                    \begin{align*}
                        \frac{\mathrm{d}}{\mathrm{d}x}\arctan x=\frac{1}{\sec^2 \theta}=\frac{1}{\cos^{-2} \theta}
                        & =\cos^2 \theta\\
                        & =\underbrace{\left(\frac{\cos\theta}{\sqrt{\sin^2 \theta+\cos^2 \theta}}\right)^2}_{\mathclap{\text{Use cosine definition and Pythagoras rule.}}}\\
                        & =\frac{1}{\tan^2 \theta+1}\\
                        & =\frac{1}{x^2+1}.
                    \end{align*}
            \end{itemize}
        \end{proof}
\end{enumerate}

\section{Exponential Functions}
\begin{enumerate}
    \item \textbf{Properties of exponents}\\
        Let \(a\) be a positive real number.
        \begin{itemize}
            \item \(a^0=1\)
            \item \(a^1=a\)
            \item \(a^m a^n=a^{m+n}\)
            \item \((a^m)^n=a^{mn}\)
            \item \(a^{m/n}=\sqrt[n]{a^m}\)
        \end{itemize}
    \item \textbf{Properties of exponential functions}\\
        \indent The function \(f(x)=a^x\) has base \(a\) for a positive real number.
        \begin{itemize}
            \item The function \(a^x\) is a continuous function.
            \item The domain of \(a^x\) is all real numbers.
            \item The range of \(a^x\) is all positive real numbers.
        \end{itemize}
    \item \textbf{The derivative of an exponential function}\\
        \indent The derivative of the exponential function is
        \[
            \frac{\mathrm{d}}{\mathrm{d}x}a^x=\frac{a^{x+\mathrm{d}x}-a^x}{\mathrm{d}x}=\frac{a^{\mathrm{d}x}-1}{\mathrm{d}x}\cdot a^x=M(a)a^x,
        \]
        where the mystery number \(M(a)\) is the slope of the tangent line at zero:
        \[
            M(a)=\left.\frac{\mathrm{d}}{\mathrm{d}x}a^x\right|_{x=0}=\frac{a^{\mathrm{d}x}-1}{\mathrm{d}x}.
        \]
    \item \textbf{Definition of \(e\)}\\
        \indent The base \(e\) is the unique real number which has \(M(e)=1\). Then
        \[
            \frac{\mathrm{d}}{\mathrm{d}x}e^x=e^x.
        \]
        Since \(M(e)\) always equals to \(1\):
        \begin{align*}
            \frac{e^{\mathrm{d}x} -1}{\mathrm{d}x} & =1\\
            e^{\mathrm{d}x} & =1+\mathrm{d}x\\
            e & =(1+\mathrm{d}x)^{\frac{1}{\mathrm{d}x}},
        \end{align*}
        Let \(\frac{1}{\mathrm{d}x}=n, n\to +\infty\). We get the formal definition of \(e\):
        \[
            e=\lim_{n\to+\infty}(1+\frac{1}{n})^n.
        \]
    \item \textbf{Differentiating exponential functions with other bases}\\
        \indent For any positive constant \(a\),
        \[
            \frac{\mathrm{d}}{\mathrm{d}x}a^x=a^x\ln a.
        \]
        \begin{proof}
            \begin{align*}
                M(a) & =\frac{a^{\mathrm{d}x}-1}{\mathrm{d}x}\\
                     & =\frac{e^{\mathrm{d}x\ln a}-1}{\mathrm{d}x}.
            \end{align*}
            Since \(\mathrm{d}x\ln a\to0\), we can rewrite the definition of \(e\) to
            \[
                e=(1+\mathrm{d}x\ln a)^\frac{1}{\mathrm{d}x\ln a}
            \]
            then
            \begin{align*}
                M(a) & =\frac{\mathrm{d}x\ln a}{\mathrm{d}x}\\
                     & =\ln a.
            \end{align*}
        \end{proof}
        \begin{proof}
            There exists another method to proof \((a^x)^\prime=a^x\ln a\) directly. It also changes to base \(e\) but later it differentiate using the chain rule.
            \begin{align*}
                \frac{\mathrm{d}a^x}{\mathrm{d}x} & =\frac{\mathrm{d}e^{x\ln a}}{\mathrm{d}x}\\
                                                 & =\frac{\mathrm{d}e^{x\ln a}}{\mathrm{d}(x\ln a)}\cdot\frac{\mathrm{d}(x\ln a)}{\mathrm{d}x}\\
                                                 & =e^{x\ln a}\ln a\\
                                                 & =a^x\ln a
            \end{align*}
        \end{proof}
\end{enumerate}

\section{Logarithms}
\begin{enumerate}
    \item \textbf{Some properties of Logarithms}\\
        \indent\(\log x\) implicitly denotes \(\log_{10}x\), which is the inverse function of \(10^x\).\par
        The natural log, denoted by \(\ln x\), is the inverse function of \(e^x\).
        \begin{itemize}
            \item \(e^{\ln x}=x\)
            \item \(\ln(ab)=\ln a+\ln b\)
            \item \(\ln a^b=b\ln a\)
        \end{itemize}
    \item \textbf{The derivative of the natural logarithm}
        \[
            \frac{\mathrm{d}}{\mathrm{d}x}\ln x=\frac{1}{x}
        \]
        \begin{proof}
            Differentiate \(e^{\ln x}=x\) with \(x\)
            \begin{align*}
                \frac{\mathrm{d}}{\mathrm{d}x}e^{\ln x} & =\frac{\mathrm{d}}{\mathrm{d}x}x\\
                \frac{\mathrm{d}e^{\ln x}}{\mathrm{d}\ln x}\cdot\frac{\mathrm{d}\ln x}{\mathrm{d}x} & =1\\
                e^{\ln x}\cdot\frac{\mathrm{d}\ln x}{\mathrm{d}x} & =1\\
                x\cdot\frac{\mathrm{d}\ln x}{\mathrm{d}x} & =1\\
                \frac{\mathrm{d}\ln x}{\mathrm{d}x} & =\frac{1}{x}.
            \end{align*}
        \end{proof}
    \item \textbf{Differentiation methods}\\
        \indent\textbf{Method 1: Change to base \(\bm{e}\)}\\
        \indent To calculate \(\left[f(x)^{g(x)}\right]^\prime\), first change base \(f(x)^{g(x)}=e^{g(x)\ln f(x)}\), then differentiate using the chain rule, and the derivatives of the exponential and natural logarithm functions.\par
        \textbf{Method 2: Logarithmic differentiation}\\
        \indent To calculate \(u^\prime\), first calculate \((\ln u)^\prime\), then use the formula \(u^\prime=u(\ln u)^\prime\) and solve for \(u^\prime\)
        \begin{proof}
            \begin{alignat*}{2}
                & & \frac{\mathrm{d}}{\mathrm{d}x}\ln u & =\frac{\mathrm{d}\ln u}{\mathrm{d}u}\cdot\frac{\mathrm{d}u}{\mathrm{d}x}\\
                & & & =\frac{1}{u}\cdot\frac{\mathrm{d}u}{\mathrm{d}x}\\
                \Leftrightarrow\, & & (\ln u)^\prime & =\frac{u^\prime}{u}\\
                \Leftrightarrow\, & & u^\prime & =u(\ln u)^\prime .
            \end{alignat*}
        \end{proof}
\end{enumerate}

