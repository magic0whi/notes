\chapter{Differentiation}
\section{Linear Approximation}
The linear approximation for a function $f$ near point $x=a$ is given by the following equivalent formulas:
\begin{gather*}
  \Delta f\approx\evalat{\frac{\dd f}{\dd x}}{x=a}\cdot\Delta x\ \text{for}\ \Delta x\ \text{near}\ 0,\\
  f(x)\approx f^\prime(a)(x-a)+f(a)\ \text{for}\ x\ \text{near}\ a.
\end{gather*}

\section{The Product Rule}
\[\text{If}\ h=f\cdot g,\ \text{then}\ h^\prime=f\cdot g^\prime+f^\prime\cdot g.\]
\begin{proof}
  To proof the product rule, we can write $f(x+\dd x)=\dd f(x)+f(x)$, the reason why this equation is important is that if we differentiate the $h=fg$:
  \begin{align*}
    \frac{\dd h}{\dd x} & =\frac\dd{\dd x}(fg)\\
                        & =\frac{f(x+\dd x)g(x+\dd x)-f(x)g(x)}{\mathrm{d}x}\\
                        & =\frac{(\dd f+f)(\dd g+g)-fg}{\dd x}\\
                        & =\frac{\dd f\dd g+g\dd f+f\dd g+\cancel{fg-fg}}{\dd x}\\
                        & =\frac{\dd f\dd g}{\dd x}+g\frac{\dd f}{\dd x}+f\frac{\dd g}{\dd x}.
  \end{align*}
  The $\frac{\dd f\dd g}{\dd x}\) can be transformed into $\frac{\dd f}{\dd x}\cdot\frac{\dd g}{\dd x}\cdot\dd x$ which would be 0 because the extra $\dd x$ at the end. Finally
  \[\frac{\dd h}{\dd x}=g\frac{\dd f}{\dd x}+f\frac{\dd g}{\dd x}.\]
\end{proof}

\section{The Quotient Rule}
\subsection{The Quotient Rule}
For functions $f$ and $g$,
\[\text{if}\ h=\frac f g\ ,\ \text{then}\ h^\prime=\frac{f^\prime g-fg^\prime}{g^2}\] at all points where $f$ and $g$ are differentiable and $g\neq0$.
\begin{proof}[The Quotient Rule]
If $h=\frac f g$, the first thing to do is to rewrite $\frac 1 g$ in the terms of a fractional power $h=fg^{-1}$. Thus the last step is a product, then we start by product rule:
    \begin{align*}
        h^\prime & =f^\prime g^{-1}+(g^{-1})^\prime f\\
                 & =\frac{f^\prime}g+(-g^{-2}g^\prime)f\\
                 & =\frac{f^\prime}g+\left(-\frac1{g^2}g^\prime\right)f\\
                 & =\frac{f^\prime}g-\frac{g^\prime f}{g^2}\\
                 & =\frac{f^\prime g+g^\prime f}{g^2}.
    \end{align*}
\end{proof}

\subsection{Derivatives of All Trigonometric Functions}
\begin{align*}
  & \frac\dd{\dd x}\sin x=\cos x\\
  & \frac\dd{\dd x}\cos x=-\sin x\\
  & \frac\dd{\dd x}\tan x=\cos^{-2}x=\sec^2 x\\
  & \frac\dd{\dd x}\cot x=-\sin^{-2}x=-\csc^2 x\\
  & \frac\dd{\dd x}\sec x=\frac{\sin x}{\cos^2 x}=\sec x\tan x\\
  & \frac\dd{\dd x}\csc x=-\frac{\cos x}{\sin^2 x}=-\csc x\cot x.
\end{align*}

\section{The Chain Rule}
For functions $f$ and $g$,
\[\text{if}\ h=f\circ g,\ \text{then}\ h^\prime=f^\prime(g)g^\prime\]
at all points where $f^\prime(g)$ and $g^\prime$ are defined.
\begin{note}Notice that $f^\prime(g)\neq[f(g)]^\prime $.\end{note}
Alternatively, let $y=f(u)$ and $u=g(x)$, then
\[
\evalat{\frac{\dd y}{\dd x}}{x=a}=\evalat{\frac{\dd y}{\dd u}}{u=g(a)}\cdot\evalat{\frac{\dd u}{\dd x}}{x=a}
\]
at any point $x=a$ where the derivatives on the right side are defined.

\section{Implicit Differentiation}
\begin{note}[Implicit functions] An implicit function is an equation involving both $x$ and $y$ (or any two variables really). you could solve for $y$ as a function of $x$, but often times that computation is messy or impossible. If a function isn't implicit, we say that it is explicit.
\end{note}

Immediately differentiating $y=x^{m/n}$ is hard. But differentiating the implicit function $y^n=x^m$ is a whole lot easier. This happens any time your function is more simply described implicitly.
\begin{align*}
  \phantom{\equiv{}} & \frac{\dd }{\dd x}(y^n=x^m)\\
            \equiv{} & ny^{n-1}\frac{\dd y}{\dd x}=mx^{m-1}\\
            \equiv{} & \frac{\dd y}{\dd x}=\frac m n\cdot\frac{x^{m-1}}{y^{n-1}}.
\end{align*}
\begin{eg}[Which of the following equations given $y$ as an implicit function of $x$?]
  \leavevmode
  \begin{itemize}
    \item$y^3=x^2$ is implicit.
    \item$y=x^{2/3}$ is \textit{explicit}.
    \item$yx=1$ is implicit.
    \item$y+x=1$ is implicit.
    \item$\sin y=\cos x$ is implicit.
  \end{itemize}
\end{eg}

\section{Inverse Function}
\subsection{Definition of Inverse Function}
If function $f$ and $g$ satisfy $g\circ f(x)=x$ and $f\circ g(y)=y$, then we say $g$ is the inverse of $f$, and denote it by $f^{-1}$ (equivalently $f=g^{-1}$). 

If a function $f$ has an inverse function $f^{-1}$, then $f^{-1}(b)=a$ if and only if $b=f(a)$.

\subsection{Definition of one-to-one (Horizontal line test)}
A function $f$ is \textbf{one-to-one} if whenever $a\neq b$, $f(a)\neq f(b)$ (Horizontal line test).

\subsection{The Inverse Trigonometry Functions}
\begin{itemize}
  \item$\sin\theta=x$ such that $\arcsin x=\theta\ \left(\theta\in\left[-\frac\pi 2,\frac\pi 2\right]\right)$;
  \item$\cos\theta=x$ such that $\arccos x=\theta\ \left(\theta\in\left[0,\pi\right]\right)$;
  \item$\tan\theta=x$ such that $\arctan x=\theta\ \left(\theta\in\left(-\frac\pi 2,\frac\pi 2\right)\right)$.
\end{itemize}
\todo{Supplement plots}
\subsection{Derivatives of Inverse Functions}
If $g$ is a (full or partial) inverse of a function $f$, then
\[g^\prime=\frac1{f^\prime\circ g}\]
at all $x$ where $f^\prime\circ g$ exists and is non-zero.

\subsection{Derivatives of Arctrig Functions}
Where defined,
\begin{itemize}
\item$\frac\dd{\dd x}\arcsin x=\frac1{\sqrt{1-x^2}}$;
\item$\frac\dd{\dd x}\arccos x=-\frac1{\sqrt{1-x^2}}$;
\item$\frac\dd{\dd x}\arctan x=\frac1{x^2+1}$.
\end{itemize}
\begin{proof}\leavevmode
  \begin{itemize}
  \item Assume $\arcsin x=\theta\ \left(\theta\in\left[-\frac\pi 2,\frac\pi 2\right]\right)$,
    \[
    \frac\dd{\dd x}\arcsin x=\frac1{\cos\theta}=\frac1{\sqrt{1-\sin^2 \theta}}=\frac1{\sqrt{1-x^2}};
    \]
  \item Assume $\arccos x=\theta\ \left(\theta\in\left[0,\pi\right]\right)$,
    \[
    \frac\dd{\dd x}\arccos x=\frac1{-\sin\theta}=\frac1{-\sqrt{1-\cos^2 \theta}}=-\frac1{\sqrt{1-x^2}};
    \]
  \item Assume $\arctan x=\theta\ \left(\theta\in\left(-\frac\pi 2,\frac\pi 2\right)\right)$,
    \begin{align*}
      \frac\dd{\dd x}\arctan x=\frac1{\sec^2 \theta}=\frac1{\cos^{-2}\theta}
      & =\cos^2 \theta\\
      & =\underbrace{\left(\frac{\cos\theta}{\sqrt{\sin^2 \theta+\cos^2 \theta}}\right)^2}_{\mathclap{\text{Cosine definition and Pythagoras rule}}}\\
      & =\frac1{\tan^2 \theta+1}=\frac1{x^2+1}.
    \end{align*}
  \end{itemize}
\end{proof}

\section{Exponential Functions}
\subsection{Properties of Exponents}
Let $a$ be a positive real number.
\begin{itemize}
\item $a^0=1$;
\item $a^1=a$;
\item $a^m a^n=a^{m+n}$;
\item $(a^m)^n=a^{mn}$;
\item $a^{m/n}=\sqrt[n]{a^m}$.
\end{itemize}

\subsection{Properties of Exponential Functions}
The function $f(x)=a^x$ has base $a$ for a positive real number.
\begin{itemize}
\item The function $a^x$ is a continuous function.
\item The domain of $a^x$ is $\R$.
\item The range of $a^x$ is $\R^+$.
\end{itemize}

\subsection{The Derivative of an Exponential Function}
The derivative of the exponential function is
\[
\frac\dd{\dd x}a^x=\frac{a^{x+\dd x}-a^x}{\dd x}=\frac{a^{\dd x}-1}{\dd x}\cdot a^x=M(a)a^x,
\]
where the mystery number $M(a)$ is the slope of the tangent line at zero:
\[M(a)=\evalat{\frac\dd{\dd x}a^x}{x=0}=\frac{a^{\dd x}-1}{\dd x}.\]

\subsection{The Definition of $\bm e$}
The base $e$ is the unique real number which has $M(e)=1$. Then
\[\frac\dd{\dd x}e^x=e^x.\]

Since $M(e)$ always equals $1$:
\begin{alignat*}{2}
         && \frac{e^{\dd x} -1}{\dd x} & =1\\
  \equiv &&                  e^{\dd x} & =1+\dd x\\
  \equiv &&                          e & =(1+\dd x)^{\frac{1}{\dd x}},
\end{alignat*}
Let $\frac1{\dd x}=n$, $n\to+\infty$, We get the formal definition of $e$:
\[e=\lim_{n\to+\infty}(1+\frac{1}{n})^n.\]

\subsection{Differentiating Exponential Functions with Other Bases}
For any positive constant $a$,
\[\frac\dd{\dd x}a^x=a^x\ln a.\]
\begin{proof}
  \begin{alignat*}{2}
         && M(a) & =\frac{a^{\dd x}-1}{\dd x}\\
  \equiv &&      & =\frac{e^{\dd x\ln a}-1}{\dd x}.
  \end{alignat*}
  Since $\dd x\ln a\to0$, we can rewrite the definition of $e$ to
  \[e=(1+\dd x\ln a)^\frac1{\dd x\ln a}\]
  then
  \[M(a)=\frac{\dd x\ln a}{\dd x}=\ln a.\]
  \hfill\qed

  There exists another method to proof $(a^x)^\prime=a^x\ln a$ directly. It also changes to base $e$ but later it differentiate using the chain rule:
  \begin{align*}
    \frac{\dd a^x}{\dd x} & =\frac{\dd e^{x\ln a}}{\dd x}\\
                          & =\frac{\dd e^{x\ln a}}{\dd(x\ln a)}\cdot\frac{\dd(x\ln a)}{\dd x}\\
                          & =e^{x\ln a}\ln a\\
                          & =a^x\ln a
  \end{align*}
\end{proof}

\todo{here}
\section{Logarithms}
\subsection{Some Properties of Logarithms}
\begin{itemize}
\item$\log x$ implicitly denotes $\log_{10}x$, which is the inverse function of $10^x$.
\item The natural log, denoted by $\ln x$, is the inverse function of $e^x$.
\item$e^{\ln x}=x$
\item$\ln(ab)=\ln a+\ln b$
\item$\ln a^b=b\ln a$
\end{itemize}
\subsection{The Derivative of the Natural Logarithm}
\[\frac\dd{\dd x}\ln x=\frac1 x\]
\begin{proof}
  Change $\ln x$ to base $e^{\ln x}$, then differentiate
  \begin{alignat*}{2}
             &&        \frac\dd{\dd x}e^{\ln x} & =\frac\dd{\dd x}x\\
    \equiv{} && \frac{\dd e^{\ln x}}{\dd\ln x}\cdot\frac{\dd\ln x}{\dd x} & =1\\
    \equiv{} && e^{\ln x}\frac{\dd\ln x}{\dd x} & =1\\
    \equiv{} &&         x\frac{\dd\ln x}{\dd x} & =1\\
    \equiv{} &&          \frac{\dd\ln x}{\dd x} & =\frac{1}{x}.
  \end{alignat*}
\end{proof}

\subsection{Differentiation Methods}
\begin{itemize}
\item Method 1: Change to base $\bm{e}$\\
To calculate $[f^g]^\prime$ for functions $f$ and $g$, first change base $f^g=e^{g\ln f}$, then differentiate and use the chain rule.
\item Method 2: Logarithmic differentiation\\
To calculate $u^\prime$, first calculate $(\ln u)^\prime$, then use the formula $u^\prime=u(\ln u)^\prime$ and solve for $u^\prime$
  \begin{proof}[$u^\prime=u(\ln u)^\prime$]
    \begin{alignat*}{2}
      && \frac{\dd }{\dd x}\ln u & =\frac{\dd \ln u}{\dd u}\cdot\frac{\dd u}{\dd x}\\
                              && & =\frac1 u\cdot\frac{\dd u}{\dd x}\\
      \equiv && (\ln u)^\prime & =\frac{u^\prime}{u}\\
      \equiv && u^\prime & =u(\ln u)^\prime .
    \end{alignat*}
  \end{proof}
\end{itemize}
