\chapter{Differentiation}
\section{Linear approximation}
\textbf{Linear approximation}

The linear approximation for a function $f$ near a point $x=a$ is given by the following equivalent formulas:
\begin{align*}
    \Delta f\approx\left.\frac{\mathrm{d}f}{\mathrm{d}x}\right|_{x=a}\cdot\Delta x & \quad\text{for $\Delta x$ near $0$,} \\
        f(x)\approx f^\prime(a)(x-a)+f(a) & \quad\text{for $x$ near $a$.}
\end{align*}

\section{Product rule}
\textbf{The product rule}

If $h(x)=f(x)g(x)$, then
$$
h^\prime (x)=f(x)g^\prime (x)+g(x)f^\prime (x)
$$

\begin{proof}
    First we need to know that the Leibniz notation
    $$
    \mathrm{d}f(x)=f(x+\mathrm{d}x)-f(x)
    $$
    which equals to $\Delta f(x)=\lim_{\Delta x\to 0} [f(x+\Delta x)-f(x)]$.

    When substitute $f(x)=x$ into both equation we get $\mathrm{d}x=(x+\mathrm{d}x)-x\text{ and }\lim_{\Delta x\to 0} (x+\Delta x)-x$.
    So
    $$
    \mathrm{d}x=\lim_{\Delta x\to 0} \Delta x,
    $$
    thus why we can replace $\Delta x$ with $\mathrm{d}x$ from the $\Delta f(x)$ and remove the limit symbol, then naturally $\Delta f(x)$ becomes $\mathrm{d}f(x)$, Perfect!

    To proof the product rule, we can write $f(x+\mathrm{d}x)=\mathrm{d}f(x)+f(x)$, the reasons why this equation is important are that if we differentiate the $h(x)=f(x)g(x)$:
    \begin{align*}
        \frac{\mathrm{d}h(x)}{\mathrm{d}x} & =\frac{\mathrm{d}}{\mathrm{d}x}[f(x)g(x)] \\
                                           & =\frac{f(x+\mathrm{d}x)g(x+\mathrm{d}x)-f(x)g(x)}{\mathrm{d}x} \\
                                           & =\frac{[\mathrm{d}f(x)+f(x)][\mathrm{d}g(x)+g(x)]-f(x)g(x)}{\mathrm{d}x} \\
                                           & =\frac{\mathrm{d}f(x)\mathrm{d}g(x)+g(x)\mathrm{d}f(x)+f(x)\mathrm{d}g(x)\bcancel{+f(x)g(x)-f(x)g(x)}}{\mathrm{d}x} \\
                                           & =\frac{\mathrm{d}f(x)\mathrm{d}g(x)}{\mathrm{d}x}+g(x)\frac{\mathrm{d}f(x)}{\mathrm{d}x}+f(x)\frac{\mathrm{d}g(x)}{\mathrm{d}x}
    \end{align*}
    The $\frac{\mathrm{d}f(x)\mathrm{d}g(x)}{\mathrm{d}x}$ can transform into $\frac{\mathrm{d}f(x)}{\mathrm{d}x}\cdot\frac{\mathrm{d}g(x)}{\mathrm{d}x}\cdot\mathrm{d}x$ which equals to $0$ because the extra $\mathrm{d}x$ at the end. Finally
    $$
    \frac{\mathrm{d}h(x)}{\mathrm{d}x}=g(x)\frac{\mathrm{d}f(x)}{\mathrm{d}x}+f(x)\frac{\mathrm{d}g(x)}{\mathrm{d}x}.
    $$
\end{proof}

\section{Quotient rule}
\begin{enumerate}
    \item \textbf{The quotient rule.} If $h(x)=\frac{f(x)}{g(x)}$ for all $x$, then
        $$
        h^\prime (x)=\frac{f^\prime (x)g(x)-f(x)g^\prime (x)}{g(x)^2}
        $$
        at all points where $f$ and $g$ are differentiable and $g(x)\neq 0$.
    \item \textbf{Derivatives of all trigonometric functions}
        \begin{align*}
            & \frac{\mathrm{d}}{\mathrm{d}x}\sin x=\cos x \\
            & \frac{\mathrm{d}}{\mathrm{d}x}\cos x=-\sin x \\
            & \frac{\mathrm{d}}{\mathrm{d}x}\tan x=\frac{1}{\cos^2 x}=\sec^2 x \\
            & \frac{\mathrm{d}}{\mathrm{d}x}\cot x=-\frac{1}{\sin^2 x}=-\csc^2 x \\
            & \frac{\mathrm{d}}{\mathrm{d}x}\sec x=\frac{\sin x}{\cos^2 x}=\sec x\tan x \\
            & \frac{\mathrm{d}}{\mathrm{d}x}\csc x=-\frac{\cos x}{\sin^2 x}=-\csc x\cot x
        \end{align*}
\end{enumerate}

\section{Chain rule}
\begin{enumerate}
    \item \textbf{The Chain Rule}

        If $h(x)=f(g(x))$, then
        $$
        h^\prime (x)=f^\prime (g(x))g^\prime (x)
        $$
        at all points where the derivatives $f^\prime (g(x))$ and $g^\prime (x)$ are defined.

        Alternatively, if $y=f(u)$ and $u=g(x)$, then
        $$
        \left.\frac{\mathrm{d}y}{\mathrm{d}x}\right|_{x=a}=\left.\frac{\mathrm{d}y}{\mathrm{d}u}\right|_{u=g(a)}\cdot\left.\frac{\mathrm{d}u}{\mathrm{d}x}\right|_{x=a}
        $$
        at any point $x=a$ where the derivatives on the right side are defined.

        \begin{proof}
            \textbf{Proof Quotient Rule by Chain Rule and Product Rule}

            If $h(x)=\frac{f(x)}{g(x)}$, transform to $h(x)=f(x)\cdot g^{-1}(x)$, the last step is a product, then we start by Product Rule
            \begin{align*}
                h^\prime (x) & =f^\prime (x)\cdot g^{-1}(x)+[g^{-1}(x)]^\prime f(x) \\
                             & =\frac{f^\prime (x)}{g(x)}+\left[[g(x)]^{-1}\right]^\prime f(x) \\
                             & =\frac{f^\prime (x)}{g(x)}+[-\frac{1}{g^2(x)}\cdot g^\prime (x)]f(x) \\
                             & =\frac{f^\prime (x)}{g(x)}-\frac{g^\prime (x)f(x)}{g^2(x)} \\
                             & =\frac{f^\prime (x)g(x)+g^\prime (x)f(x)}{g^2(x)}
            \end{align*}
        \end{proof}
\end{enumerate}
