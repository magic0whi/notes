\chapter{Integration Theory}
\section{The Definite Integral}
\subsection{Geometric definition of the definite integral}
The \textbf{definite integral of $f$ from $a$ to $b$}, denoted by
\[\int_a^bf(x)\dd x,\]
is the area of the region above the $x$-axis, below the curve $y=f(x)$, and in between the two vertical lines $x=a$ and $x=b$, as shown shaded in the figure below.

The $x$-values $a$ and $b$ are called the \textbf{lower} and \textbf{upper limits of the integral}. (This is a different sense of the word ``limit'' from when we take the limit of a function.)
\todo{Supplement figure}

The only difference between the notation for definite and indefinite integrals is that definite integrals have limits but indefinite integrals do not.
\subsection{Summation notation}
The $\sum$ notation in a compact way to denote a sum in which each term is obtained from a formula:
\[\sum_{i=1}^na_i=a_1+a_2+\cdots+a_{n-1}+a_n\]
where $i$ indexes the terms, and $a_i$ is a formula for the $i$\textsuperscript{th} term of the sum.

The notation $\sum_{i=1}^na_i$ reads ``the sum of $a_i$ from $i=1$ to $i=n$.''

For example, if $a_i=i^2$, that is, the formula for the term indexed by $i$ is $i^2$, then
\[\sum_{i=1}^ni^2=1^2+2^2+3^2+\cdots+(n-1)^2+n^2\]
\subsection{Riemann sums}
Let us summarize in precise terms the steps for evaluating $\int_a^bf(x)\dd x$ using a Riemann Sum.
\begin{enumerate}
\item Divide $[a,b]$ into $n$ equal subintervals.
  \todo{Supplement figure}\\
  Then each interval is of length
  \[\Delta x=\frac{b-a}n.\]
  Let the $i$\textsuperscript{th} subinterval be the \textbf{base} of $i$\textsuperscript{th} rectangle.
\item Choose a point $c_i$ within the $i$\textsuperscript{th} subinterval. Choose $f(c_i)$ be the \textbf{height} of the $i$\textsuperscript{th} rectangle.\todo{Supplement figure}
\item Add up the areas of the $n$ rectangles. The total area of $n$ rectangles is:
  \[\underbrace{f(c_1)}_{\text{height}}\underbrace{\Delta x}_{\text{base}}
  +\underbrace{f(c_2)}_{\text{height}}\underbrace{\Delta x}_{\text{base}}
  +\cdots
  +\underbrace{f(c_n)}_{\text{height}}\underbrace{\Delta x}_{\text{base}}
  =\sum_{i=1}^nf(c_i)\Delta x
  \]
\item Take the limit as the rectangles become infinitesimally thin, ($\Delta x\to0$, or equivalent $n\to\infty$). This limit is the actual area under the curve between $a$ and $b$.
  \[\lim_{n\to\infty}\sum_{i=1}^nf(c_i)\Delta x=\int_a^bf(x)\dd x\]
  The sum of the areas of the $n$ rectangles, $\sum_{i=1}^nf(c_i)\Delta x$, is called a \textbf{Riemann Sum}. If we pick $c_i$ to be the left endpoint of the $i$\textsuperscript{th} subinterval, the Riemann sum is called a \textbf{left Riemann Sum}. Similarly, if $c_i$ is the right endpoint of the $i$\textsuperscript{th} interval, the Riemann sum is called a \textbf{right Riemann Sum}.

  However, in the limit $n\to\infty$ (so that $\Delta x\to0$), this distinction is no longer needed. The limit of any Riemann Sum, no matter what the $c_i$'s within the subinterval are, is equal to the exact area under the curve.
\end{enumerate}
\subsection{Cumulative sums}
We have been defining the definite integral $\int_a^bf(x)\dd x$ geometrically as the area under a curve. But $f(x)$ and $x$ can represent quantities other than length.

For instance, if $x$ represents time, with units of hours, and $f(x)$ represents velocity, with units of \unit[per-mode=single-symbol]{\km\per\hour}. So the units of the integral, $\int_a^bf(x)\dd x$, is \unit{\km}, which are the units of distance. But the integral $\int_a^bf(x)\dd x$ still represents the area under the curve, so the units of this area are \unit{\km}.

Most applications of the integral will involve seeing it as a cumulative sum.
\section{First Fundamental Theorem of Calculus}
\subsection{The First Fundamental Theorem of Calculus}
The \textbf{First Fundamental Theorem of Calculus} states that:

If $F$ is differentiable, and $F^\prime=f$ is continuous, then
\[\int_a^bf(x)\dd x=F(b)-F(a)=F(x)=\evalat[b]{F(x)}{a}.\]

In other words, the definite integral of a function is the difference between the values of its antiderivative at the limits of the definite integral.

We will abbreviate the \textbf{First Fundamental Theorem of Calculus} as \textbf{FTC1}.

The FTC1 connects the definite integral to the antiderivative. With this connection, we can now compute definite integrals using antiderivatives, and dispense with Riemann sums.
TODO
