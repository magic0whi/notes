\chapter{Approximations}\todo{Progress is here}
\section{Linear approximation: measurement error}
\begin{enumerate}
    \item \textbf{Linear Approximations near $\bm{x=0}$}

        Recall that the linearization of a function $f(x)$ near $x=0$ is given by the formula $f(x)\approx f(0)+f^\prime (0)x$.
        \begin{itemize}
            \item $(1+x)^n\approx 1+nx$
            \item $\sin x\approx x$
            \item $\cos x\approx 1$
            \item $e^x\approx x+1$
            \item $\ln(1+x)\approx x$
        \end{itemize}
    \item \textbf{Approximations (likely nonlinear) of compositions}

        Suppose $g(x)$ is a function such that $g(0)=0$. To find an approximation of a function $f(g(x))$ near $x=0$, we can take a linear approximation for $f(u)$ and then substitute $g(x)$ in for $u$. The resulting approximation is likely nonlinear, but it is still an approximation.

        Warning: this only works if $g(0)=0$.
        $$
            f(u)\approx f(0)+f^\prime (u)u\Rightarrow f(g(x))\approx f(0)+f^\prime (0)g(x)
        $$
    \item \textbf{Linear approximations of products}

        To find the linear approximation of a function $h(x)=f(x)g(x)$ near $x=0$, it suffices to find a linear approximation for $f(x)$, find a linear approximation for $g(x)$, and then the linear approximation for $h(x)$ is the product of these two approximations where we cancel all of the terms that are quadratic (or higher for more products).
		\begin{align*}
            h(x) & \approx (f(0)+f^\prime (0)x)(g(0)+g^\prime (0)x) \\
                 & =f(0)g(0)+(f^\prime (0)g(0)+f(0)g^\prime (0))x+f^\prime (0)g^\prime (0)x^2 \\
                 & \approx f(0)g(0)+(f^\prime (0)g(0)+f(0)g^\prime (0))x
        \end{align*}
\end{enumerate}
\section{Quadratic approximation}
\begin{enumerate}
    \item \textbf{Best fit quadratic}

        The bast fit quadratic to a function $f(x)$ at the point $x=0$ is the quadratic function $q(x)$ whose value agrees with the value of $f$ at $x=0$, and whose first and second derivatives agree with the first and second derivatives of $f$ at $x=0$, i.e.:
        \begin{align*}
            f(0) & =q(0) \\
            f^\prime (0) & =q^\prime (0) \\
            f^{\prime\prime}(0) & =q^{\prime\prime}(0)
        \end{align*}
    \item \textbf{Quadratic Approximation}

        The \textbf{quadratic approximation} near $x=a$ is the \textbf{best fit parabola} to $f$ at the point $x=a$.

        The formula for the quadratic approximation of a function $f$ near a point $x=a$ is 
        $$
        f(x)\approx f(a)+f^\prime (a)(x-a)+\frac{f^{\prime\prime}(a)}{2}(x-a)^2,
        $$
        When $a=0$, this quadratic approximation becomes
        $$
        f(x)\approx f(0)+f^\prime (0)x+\frac{f^{\prime\prime}(0)}{2}x^2.
        $$
    \item \textbf{Big-O notation}

        A function $f(x)$ is on the order $x^n$ near $x=0$, which is denoted using big ``O'' notation as $f(x)=O(x^n)$ near $x=0$ if $|f(x)|\leq kx^n$.
    \item \textbf{Library of quadratic approximations}
        \begin{itemize}
            \item $e^x=1+x+\frac{x^2}{2}+O(x^3)$
            \item $\sin(x)=x+O(x^3)$
            \item $\cos(x)=1-\frac{x^2}{2}+O(x^3)$
            \item $\ln(1+x)=x-\frac{x^2}{2}+O(x^3)$
            \item $(1+x)^r=1+rx+\frac{r(r-1)}{2}x^2+O(x^3)$
        \end{itemize}
\end{enumerate}

\section{Newton's Method}
\begin{figure}[H]
    \centering
    \incfig{3approximations_3.1}
    \caption{Newton's Method}
\end{figure}
Given a function $f(x)$, find $x$ such that $f(x)=0$.
\begin{enumerate}
    \item Make a good guess $x_0$.
    \item Call $x_1$ the $x$-intercept of the tangent line through $(x_0, f(x_0))$. It has the formula.
        $$
        \boxed{x_1=x_0-\frac{f(x_0)}{f^\prime (x_0)}.}
        $$
    \item Repeat. The general formula is
        $$
        \boxed{x_{n+1}=x_n-\frac{f(x_n)}{f^\prime (x_n)}}
        $$
        for $n=0,1,2,\cdots$.
\end{enumerate}
