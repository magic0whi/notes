\chapter{Approximations}
\section{Linear Approximation: Measurement Error}
\subsection{Linear Approximations near $\bm{x=0}$}
\begin{itemize}
\item$(1+x)^n\approx1+nx$
\item$\sin x\approx x$
\item$\cos x\approx1$
\item$e^x\approx x+1$
\item$\ln(1+x)\approx x$
\end{itemize}
\begin{note}[Recall]
The linearization of a function $f$ near $x=a$ is given by the formula $f(x)\approx f(a)+f^\prime (a)(x-a)$.
\end{note}

\subsection{Approximations of Compositions}
Suppose $g$ is a function such that $\bm{g(0)=0}$. To find an approximation of a function $f\circ g(x)$ \textbf{near} $\bm{x=0}$, we can take a linear approximation for $f(u)$ and then substitute $g(x)$ in for $u$. The resulting approximation is likely nonlinear, but i's still an approximation.
\[f(u)\approx f(0)+f^\prime(0)u\to f[g(x)]\approx f(0)+f^\prime(0)g(x)\]

\subsection{Linear Approximations of Products}
For functions $f$ and $g$, to find the linear approximation of a function $h=fg$ \textbf{near} $\bm{x=0}$, it suffices to find a linear approximation of $f$ and $g$ separately, then the linear approximation of $h$ is the product of these two approximations where we ignore all the terms that are quadratic or higher.
\begin{align*}
  h(x) & \approx[f(0)+f^\prime(0)x][g(0)+g^\prime(0)x]\\
  & =f(0)g(0)+[f(0)g^\prime(0)+f^\prime(0)g(0)]x+f^\prime(0)g^\prime(0)x^2\\
  & \approx f(0)g(0)+[f(0)g^\prime (0)+f^\prime (0)g(0)]x
\end{align*}

\section{Quadratic Approximation}
\subsection{Best Fit Quadratic}
The best fit quadratic curve to a function $f$ at the point $x=0$ is the quadratic function $q$ whose value agrees with the value of $f$ at $x=0$, and whose first and second derivatives agree with the first and second derivatives of $f$ at $x=0$:
\begin{gather*}
  f(0)=q(0)\\
  f^\prime (0)=q^\prime (0)\\
  f^{\prime\prime}(0)=q^{\prime\prime}(0)
\end{gather*}

\subsection{Quadratic Approximation}
The quadratic approximation near $x=a$ is the \textit{best fit parabola} to $f$ at the point $x=a$, which is 
\[f(x)\approx f(a)+f^\prime (a)(x-a)+\frac{f^{\prime\prime}(a)}{2}(x-a)^2.\]

When $a=0$, this formula becomes
\[f(x)\approx f(0)+f^\prime (0)x+\frac{f^{\prime\prime}(0)}2 x^2.\]

\subsection{Big-O notation}
A function $f$ on the order $x^n$ near $x=0$ can be denoted using big ``O'' notation as $f(x)=O(x^n)$ near $x=0$ if $\abs{f(x)}\leq kx^n$ (where $k$ is a constant).

\subsection{Some Quadratic Approximations}
\begin{itemize}
\item$e^x=1+x+\frac{x^2}2+O(x^3)$
\item$\sin x=x+O(x^3)$
\item$\cos x=1-\frac{x^2}2+O(x^3)$
\item$\ln(1+x)=x-\frac{x^2}2+O(x^3)$
\item$(1+x)^n=1+rx+x^2\frac{n(n-1)}2+O(x^3)$
\end{itemize}

\section{Newton's Method}
\begin{figure}[H]
  \centering
  \incfig{3approximations_3.1}
  \caption{Newton's Method}
\end{figure}
Given a function $f(x)$, find $x$ such that $f(x)=0$.
\begin{enumerate}
\item Make a good guess $x_0$;
\item Name $x_1$ the $x$-intercept of the tangent line through $(x_0,f(x_0))$, which has the form
  \[\boxed{x_1=x_0-\frac{f(x_0)}{f^\prime(x_0)};}\]
  \begin{proof}
    Plug in $(x_1, 0)$ to the tangent line through $(x_0, f(x_0))$ is $y=f(x_0)+f^\prime(x_0)(x-x_0)$ will gets $x_1=x_0-\frac{f(x_0)}{f^\prime(x_0)}$.
  \end{proof}
\item Repeat, the general formula is
  \[\boxed{x_{n+1}=x_n-\frac{f(x_n)}{f^\prime(x_n)};}\]
  for $n=0,1,2,\cdots$.
\end{enumerate}
