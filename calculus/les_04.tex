\chapter{Applications}
\section{Graphing and critical points: 1st and 2nd derivative tests}
\begin{enumerate}
    \item \textbf{The First Derivative Test - Finding Local Maxima and Minima}\\
        \indent Suppose the function \(f(x)\) is continuous at \(x=a\) and has a critical point at \(x=a\). \\
        \begin{itemize}
            \item \(f\) has a local minimum at \(x=a\) if \(f^\prime (x)<0\) just to the left of \(a\) and \(f^\prime (x)>0\) just to the right of \(a\).
            \item \(f\) has a local maximum at \(x=a\) if \(f^\prime (x)>0\) just to the left of \(a\) and \(f^\prime (x)>0\) just to the right of \(a\).
            \item The point \(x=a\) is neither a local minimum nor a local maximum of \(f\) if \(f^\prime (x)\) has the same sign just to the left of \(a\) and just to the right of \(a\).
        \end{itemize}
        \begin{note}
            \textbf{Just to the left or right}\\
            When we use the phrase ``\(f^\prime (x)>0\) just to the left of \(a\)'', we mean that there is some open interval \((a-c,a)\) of positive width \(c\) on which \(f^\prime\) is positive. This interval does not have to be very big, as long as it has some size.\\
            Similarly, ``\(f^\prime (x)>0\)'' just to the right of \(a\)'' means that there is some open interval \((a,a+d)\) of positive width \(d\) on which \(f^\prime\) is positive.
        \end{note}
    \item \textbf{The Second Derivative Test}\\
        \indent Suppose that \(x=a\) is a critical point of \(f\), with \(f^\prime (a)=0\).
\end{enumerate}
