\chapter{Applications}
\section{Graphing and Critical Points: 1st and 2nd Derivative Tests}
\subsection{The First Derivative Test\textemdash Finding Local Maxima and Minima}
Suppose the function $f$ is continuous at $x=a$ and has a critical point at $x=a$.
\begin{itemize}
\item $f$ has a local minimum at $x=a$ if $f^\prime(x)<0$ just to the left of $a$ and $f^\prime(x)>0$ just to the right of $a$.
  \begin{figure}[H]
    \centering
    \incfig{4applications_1.1}
    \caption{local minimum}
  \end{figure}
\item $f$ has a local maximum at $x=a$ if $f^\prime (x)>0$ just to the left of $a$ and $f^\prime (x)>0$ just to the right of $a$.
  \begin{figure}[H]
    \centering
    \incfig{4applications_1.2}
    \caption{local maximum}
  \end{figure}
\item The point $x=a$ is neither a local minimum nor a local maximum of $f$ if $f^\prime (x)$ has the same sign just to the left and right of $a$.
\end{itemize}

\subsection{The Second Derivative Test}
Suppose that $x=a$ is a critical point of $f$, with $f^\prime(a)=0$.
\begin{itemize}
\item If $f^{\prime\prime}(a)>0$, then $f$ has a local minimum at $x=a$.
\item If $f^{\prime\prime}(a)<0$, then $f$ has a local maximum at $x=a$.
\item If $f^{\prime\prime}(a)=0$ or does not exist, then the test is inconclusive\textemdash there might be a local minimum, or a local maximum, or neither.
\end{itemize}
\begin{definition}[Critical points]
Critical points are where $f^\prime=0$ or where $f$ is not differentiable.
\end{definition}
\begin{figure}[H]
  \centering
  \incfig{4applications_1.3}
  \caption{Second Derivative Test}
\end{figure}
\begin{note}
  In cases where the first derivative test is inconclusive, we can try the third derivative to determine whether critical point is a local minimum or maximum. But in most cases use interval sampling of first derivative is fewer efforts.
\end{note}

\section{The Big Picture: Limits and Asymptotics}
\subsection{Curve Sketching}
\begin{enumerate}
\item Plot
  \begin{itemize}
  \item discontinuities;
  \item end points ($x\to\pm\infty$);
  \item easy points ($x=0$ or $y=0$).
  \end{itemize}
\item Plot critical points and values (Try to solve $f^\prime=0$).
\item Decide whether $f^\prime<0$ or $f^\prime>0$ on each interval between discontinuities, end points, and critical points.
\item Identify where $f^{\prime\prime}<0$ (Concave down) and $f^{\prime\prime}>0$ (Concave up). Identify inflection points.
\end{enumerate}

\subsection{L'H\^opital's Rule}
\begin{itemize}
\item (Version 1) Indeterminate form $0/0$\\
  If functions $f(x)\to0$, $g(x)\to0$ as $x\to a$, and $f$, $g$ are differentiable near the point $x=a$, then limit
  \[\lim_{x\to a}\frac{f(x)}{g(x)}=\lim_{x\to a}\frac{f^\prime(x)}{g^\prime(x)}\]
  provided that the right-hand limit exists or is $\pm\infty$.
\item (Version 2) Indeterminate form $\infty/\infty$\\
  If functions $f(x)\to\infty$, $g(x)\to\infty$ as $x\to a$, and $f$, $g$ are differentiable near the point $x=a$, then limit
  \[\lim_{x\to a}\frac{f(x)}{g(x)}=\lim_{x\to a}\frac{f^\prime(x)}{g^\prime(x)}\]
  provided that the right-hand limit exists or is $\pm\infty$.
\item Other indeterminate forms $0\cdot\infty$, $\infty-\infty$, $0^0$, $1^\infty$ and $\infty^0$ should be rearranged to be of the form $0/0$ or $\infty/\infty$ in order to apply L'H\^opital's rule (By using differentiation methods like change base to $e$).
\end{itemize}
\begin{note}
  \begin{itemize}\leavevmode
  \item We can replace $a$ with $\pm\infty$; or $a^+$, $a^-$ and the results still hold.
  \item It's worth to look again before L'Hop. Apply L'Hop to non-indeterminate form is probably got a wrong result.
  \end{itemize}
\end{note}
\todo{\footnotesize General proof after learned something like $\inf$ and $\sup$ in the order theory}
\todo{\scriptsize Review: Order theory is mentioned in discrete mathematics, but I wrote it in Chinese, rewrite is required to mapping verbs into English}

\section{Optimization: Max/Min Problems}
\subsection{The Extreme Value Theorem}
If function $f$ is continuous on a finite closed interval $[a,b]$, then there must exist points at which $f$ attains its maximum and minimum on $[a,b]$.

\subsection{Candidates for Extrema}

The maximum and minimum can only be attained at critical points or endpoints. So we just need to run through all of those candidates to find the largest and smallest values of $f$.
\begin{note}[Strategy for Optimization Problems]\leavevmode
  \begin{enumerate}
  \item Draw a picture;
  \item Name the independent variable;
  \item Write quantity (equation) we want to optimize, solely in terms of that variable.
  \item Determine the allowable interval;
  \item Check critical points and endpoints.
  \end{enumerate}
\end{note}
\begin{note}
The Extreme Value Theorem guarantees to attain of maximum or minimum value.
\end{note}

\section{Related rates}
\subsection{Related Rates Strategy}
\begin{enumerate}
\item Start with a good picture;
\item Identify the relevant variables and rates;
\item Find an equation relating the relevant variables that always holds;
\item Differentiate implicitly;
\item Plug in and solve.
\end{enumerate}
