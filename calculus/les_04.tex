\chapter{Applications}
\section{Graphing and Critical Points: 1st and 2nd Derivative Tests}
\subsection{The First Derivative Test\textemdash Finding Local Maxima and Minima}
Suppose the function $f$ is continuous at $x=a$ and has a critical point at $x=a$.
\begin{itemize}
\item $f$ has a local minimum at $x=a$ if $f^\prime(x)<0$ just to the left of $a$ and $f^\prime(x)>0$ just to the right of $a$.
  \begin{figure}[H]
    \centering
    \incfig{4applications_1.1}
    \caption{local minimum}
  \end{figure}
\item $f$ has a local maximum at $x=a$ if $f^\prime (x)>0$ just to the left of $a$ and $f^\prime (x)>0$ just to the right of $a$.
  \begin{figure}[H]
    \centering
    \incfig{4applications_1.2}
    \caption{local maximum}
  \end{figure}
\item The point $x=a$ is neither a local minimum nor a local maximum of $f$ if $f^\prime (x)$ has the same sign just to the left and right of $a$.
\end{itemize}

\subsection{The Second Derivative Test}
Suppose that $x=a$ is a critical point of $f$, with $f^\prime(a)=0$.
\begin{itemize}
\item If $f^{\prime\prime}(a)>0$, then $f$ has a local minimum at $x=a$.
\item If $f^{\prime\prime}(a)<0$, then $f$ has a local maximum at $x=a$.
\item If $f^{\prime\prime}(a)=0$ or does not exist, then the test is inconclusive\textemdash there might be a local minimum, or a local maximum, or neither.
\end{itemize}
\begin{figure}[H]
  \centering
  \incfig{4applications_1.3}
  \caption{Second Derivative Test}
\end{figure}
\begin{note}
  In cases where the first derivative test is inconclusive, we can try the third derivative to determine whether critical point is a local minimum or maximum. But in most cases use interval sampling of first derivative is less efforts.
\end{note}
